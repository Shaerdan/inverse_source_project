\documentclass[11pt,a4paper]{article}

% ============================================================================
% PACKAGES
% ============================================================================
\usepackage[utf8]{inputenc}
\usepackage[T1]{fontenc}
\usepackage{amsmath,amssymb,amsthm}
\usepackage{mathtools}
\usepackage{bm}
\usepackage{geometry}
\usepackage{graphicx}
\usepackage{float}
\usepackage{booktabs}
\usepackage{array}
\usepackage{algorithm}
\usepackage{algpseudocode}
\usepackage{listings}
\usepackage{xcolor}
\usepackage{hyperref}
\usepackage{cleveref}
\usepackage{natbib}
\usepackage{caption}
\usepackage{subcaption}
\usepackage{tikz}
\usetikzlibrary{shapes,arrows,positioning}

% ============================================================================
% PAGE SETUP
% ============================================================================
\geometry{margin=1in}
\setlength{\parindent}{0pt}
\setlength{\parskip}{6pt}

% ============================================================================
% THEOREM ENVIRONMENTS
% ============================================================================
\theoremstyle{plain}
\newtheorem{theorem}{Theorem}[section]
\newtheorem{lemma}[theorem]{Lemma}
\newtheorem{proposition}[theorem]{Proposition}
\newtheorem{corollary}[theorem]{Corollary}

\theoremstyle{definition}
\newtheorem{definition}[theorem]{Definition}
\newtheorem{example}[theorem]{Example}
\newtheorem{remark}[theorem]{Remark}
\newtheorem{problem}{Problem}[section]

% ============================================================================
% CUSTOM COMMANDS
% ============================================================================
\newcommand{\R}{\mathbb{R}}
\newcommand{\C}{\mathbb{C}}
\newcommand{\N}{\mathbb{N}}
\newcommand{\boundary}{\partial\Omega}
\newcommand{\inner}[2]{\langle #1, #2 \rangle}
\newcommand{\norm}[1]{\left\| #1 \right\|}
\newcommand{\abs}[1]{\left| #1 \right|}
\newcommand{\dd}{\mathrm{d}}
\newcommand{\pp}{\partial}
\newcommand{\grad}{\nabla}
\newcommand{\divergence}{\nabla \cdot}
\newcommand{\laplacian}{\Delta}
\newcommand{\vect}[1]{\mathbf{#1}}
\newcommand{\mat}[1]{\mathbf{#1}}
\newcommand{\dirac}{\delta}
\newcommand{\argmin}{\operatorname{argmin}}

% Code listing style
\lstset{
    basicstyle=\small\ttfamily,
    keywordstyle=\color{blue},
    commentstyle=\color{gray},
    stringstyle=\color{red},
    numbers=left,
    numberstyle=\tiny,
    numbersep=5pt,
    frame=single,
    breaklines=true
}

% ============================================================================
% DOCUMENT
% ============================================================================
\begin{document}

% ============================================================================
% TITLE PAGE
% ============================================================================
\begin{titlepage}
    \centering
    \vspace*{2cm}
    
    {\Huge\bfseries Inverse Source Localization\\[0.5cm] for the Poisson Equation}
    
    \vspace{1cm}
    
    {\Large Mathematical Foundations and Numerical Methods}
    
    \vspace{2cm}
    
    {\large Technical Report}
    
    \vspace{1cm}
    
    {\large December 2025}
    
    \vfill
    
    \begin{abstract}
    \noindent
    This report presents a comprehensive mathematical framework for inverse source localization problems governed by the Poisson equation with Neumann boundary conditions. We develop both the forward problem---computing boundary measurements from interior point sources---and the inverse problem---recovering source locations and intensities from boundary data. Three complementary numerical approaches are presented and compared: (1) an \emph{analytical solver} using the closed-form Neumann Green's function for the unit disk derived via the method of images, (2) a \emph{Boundary Element Method} (BEM) with numerical integration for validation and general boundary data, and (3) a \emph{Finite Element Method} (FEM) for arbitrary meshed domains. Additionally, conformal mapping techniques extend the analytical approach to general simply connected domains. For the inverse problem, we formulate both nonlinear optimization (for continuous source positions) and linear algebraic approaches (for discretized source grids) with various regularization strategies including Tikhonov ($L^2$), sparsity-promoting ($L^1$), and Total Variation (TV). Numerical experiments reveal that the choice of forward solver significantly affects the optimization landscape for nonlinear inverse problems. Complete algorithmic details and implementation considerations are provided.
    \end{abstract}
    
\end{titlepage}

% ============================================================================
% TABLE OF CONTENTS
% ============================================================================
\tableofcontents
\newpage

% ============================================================================
% SECTION 1: INTRODUCTION
% ============================================================================
\section{Introduction}
\label{sec:introduction}

The inverse source problem for elliptic partial differential equations arises in numerous applications including geophysical prospecting \citep{isakov2006inverse}, medical imaging \citep{ammari2004reconstruction}, environmental monitoring \citep{el2005identification}, and non-destructive testing \citep{andrieux2006inverse}. The fundamental challenge is to determine the location and strength of interior sources from measurements taken only on the boundary of the domain.

\subsection{Problem Overview}

Consider a bounded domain $\Omega \subset \R^2$ with boundary $\boundary$. The forward problem consists of solving the Poisson equation with point sources:
\begin{equation}
    \label{eq:forward_intro}
    -\laplacian u = \sum_{k=1}^{N} q_k \dirac(\vect{x} - \bm{\xi}_k) \quad \text{in } \Omega
\end{equation}
subject to appropriate boundary conditions, where $\bm{\xi}_k \in \Omega$ are source locations and $q_k \in \R$ are source intensities.

The inverse problem seeks to recover the source parameters $\{(\bm{\xi}_k, q_k)\}_{k=1}^{N}$ from boundary measurements of $u$ or its derivatives.

\subsection{Scope and Organization}

This report is organized as follows:
\begin{itemize}
    \item \textbf{Section~\ref{sec:forward}}: Mathematical formulation of the forward problem
    \item \textbf{Section~\ref{sec:analytical}}: Analytical solution for the unit disk
    \item \textbf{Section~\ref{sec:bem}}: Boundary Element Method with numerical integration
    \item \textbf{Section~\ref{sec:fem}}: Finite Element Method discretization
    \item \textbf{Section~\ref{sec:conformal}}: Conformal mapping for general domains
    \item \textbf{Section~\ref{sec:inverse}}: Inverse problem formulations
    \item \textbf{Section~\ref{sec:regularization}}: Regularization methods
    \item \textbf{Section~\ref{sec:numerical}}: Numerical experiments and comparison
    \item \textbf{Section~\ref{sec:implementation}}: Implementation details
\end{itemize}


% ============================================================================
% SECTION 2: FORWARD PROBLEM
% ============================================================================
\section{The Forward Problem}
\label{sec:forward}

\subsection{Strong Formulation}

Let $\Omega \subset \R^2$ be a bounded, simply connected domain with smooth boundary $\boundary$. We consider the Poisson equation with point sources and homogeneous Neumann boundary conditions:

\begin{problem}[Strong Form]
\label{prob:strong}
Find $u: \Omega \to \R$ such that
\begin{align}
    -\laplacian u &= f \quad \text{in } \Omega \label{eq:poisson}\\
    \frac{\pp u}{\pp \vect{n}} &= 0 \quad \text{on } \boundary \label{eq:neumann}
\end{align}
where $\vect{n}$ is the outward unit normal to $\boundary$, and the source term is
\begin{equation}
    \label{eq:source_term}
    f(\vect{x}) = \sum_{k=1}^{N} q_k \dirac(\vect{x} - \bm{\xi}_k)
\end{equation}
with source locations $\bm{\xi}_k \in \Omega$ and intensities $q_k \in \R$.
\end{problem}

\subsection{Compatibility Condition}

For the Neumann problem to admit a solution, the source term must satisfy a compatibility condition. Integrating \eqref{eq:poisson} over $\Omega$ and applying the divergence theorem:
\begin{equation}
    -\int_\Omega \laplacian u \, \dd\vect{x} = -\int_{\boundary} \frac{\pp u}{\pp \vect{n}} \, \dd s = 0
\end{equation}

Thus, we require:
\begin{equation}
    \label{eq:compatibility}
    \int_\Omega f \, \dd\vect{x} = \sum_{k=1}^{N} q_k = 0
\end{equation}

\begin{remark}[Physical Interpretation]
The compatibility condition \eqref{eq:compatibility} states that the total source strength must equal the total sink strength. This reflects conservation: with no flux through the boundary, what flows out of sources must flow into sinks.
\end{remark}

\subsection{Uniqueness}

The solution to Problem~\ref{prob:strong} is unique only up to an additive constant. We fix this ambiguity by imposing:
\begin{equation}
    \label{eq:zero_mean}
    \int_\Omega u \, \dd\vect{x} = 0
\end{equation}

\subsection{Weak Formulation}

The presence of Dirac delta distributions in \eqref{eq:source_term} requires a weak (variational) formulation. Let $H^1(\Omega)$ denote the Sobolev space of functions with square-integrable weak derivatives.

Multiplying \eqref{eq:poisson} by a test function $v \in H^1(\Omega)$ and integrating by parts:
\begin{align}
    \int_\Omega (-\laplacian u) v \, \dd\vect{x} &= \int_\Omega f v \, \dd\vect{x} \\
    \int_\Omega \grad u \cdot \grad v \, \dd\vect{x} - \int_{\boundary} \frac{\pp u}{\pp \vect{n}} v \, \dd s &= \int_\Omega f v \, \dd\vect{x}
\end{align}

With the Neumann condition \eqref{eq:neumann}, the boundary integral vanishes.

\begin{problem}[Weak Form]
\label{prob:weak}
Find $u \in H^1(\Omega)$ with $\int_\Omega u \, \dd\vect{x} = 0$ such that for all $v \in H^1(\Omega)$:
\begin{equation}
    \label{eq:weak_form}
    \int_\Omega \grad u \cdot \grad v \, \dd\vect{x} = \sum_{k=1}^{N} q_k \, v(\bm{\xi}_k)
\end{equation}
\end{problem}

\subsection{Green's Function Representation}

The solution to Problem~\ref{prob:strong} can be expressed using the Neumann Green's function.

\begin{definition}[Neumann Green's Function]
\label{def:greens}
The Neumann Green's function $G: \Omega \times \Omega \to \R$ satisfies:
\begin{align}
    -\laplacian_{\vect{x}} G(\vect{x}, \bm{\xi}) &= \dirac(\vect{x} - \bm{\xi}) - \frac{1}{|\Omega|} \quad \text{in } \Omega \label{eq:greens_pde}\\
    \frac{\pp G}{\pp \vect{n}_{\vect{x}}} &= 0 \quad \text{on } \boundary \label{eq:greens_bc}\\
    \int_\Omega G(\vect{x}, \bm{\xi}) \, \dd\vect{x} &= 0 \label{eq:greens_norm}
\end{align}
\end{definition}

The solution to Problem~\ref{prob:strong} is:
\begin{equation}
    \label{eq:greens_representation}
    u(\vect{x}) = \sum_{k=1}^{N} q_k \, G(\vect{x}, \bm{\xi}_k)
\end{equation}


% ============================================================================
% SECTION 3: ANALYTICAL SOLUTION
% ============================================================================
\section{Analytical Solution for the Unit Disk}
\label{sec:analytical}

For the unit disk $D = \{\vect{x} \in \R^2 : |\vect{x}| < 1\}$, the Neumann Green's function admits an explicit closed-form expression using the method of images \citep{jackson1999classical,stakgold2011greens}.

\subsection{The Method of Images}

The key idea is to place an ``image'' source outside the domain such that the combined potential from the real source and its image satisfies the Neumann boundary condition.

\begin{theorem}[Green's Function for Unit Disk]
\label{thm:disk_greens}
For the unit disk with homogeneous Neumann boundary conditions, the Green's function is:
\begin{equation}
    \label{eq:disk_greens}
    G_D(\vect{x}, \bm{\xi}) = -\frac{1}{2\pi}\left[\ln|\vect{x} - \bm{\xi}| + \ln|\vect{x} - \bm{\xi}^*| - \ln|\bm{\xi}|\right]
\end{equation}
where $\bm{\xi}^* = \bm{\xi}/|\bm{\xi}|^2$ is the Kelvin transform (image point) of $\bm{\xi}$.
\end{theorem}

\begin{proof}
We verify that \eqref{eq:disk_greens} satisfies the required conditions.

\textbf{(i) PDE in the interior}: The fundamental solution $\Phi(\vect{x}, \bm{\xi}) = -\frac{1}{2\pi}\ln|\vect{x} - \bm{\xi}|$ satisfies $-\laplacian \Phi = \dirac(\vect{x} - \bm{\xi})$. Since $\bm{\xi}^*$ lies outside $D$ (for $|\bm{\xi}| < 1$, we have $|\bm{\xi}^*| = 1/|\bm{\xi}| > 1$), the image term is harmonic in $D$.

\textbf{(ii) Neumann condition}: For $\vect{x} \in \pp D$ (i.e., $|\vect{x}| = 1$), the normal derivatives of the source and image terms cancel on the boundary due to the reflection property of the Kelvin transform.
\end{proof}

\subsection{Analytical Forward Solver}

Using \eqref{eq:disk_greens}, the forward problem has the explicit solution:
\begin{equation}
    \label{eq:analytical_forward}
    u(\vect{x}) = -\frac{1}{2\pi} \sum_{k=1}^{N} q_k \left[\ln|\vect{x} - \bm{\xi}_k| + \ln|\vect{x} - \bm{\xi}_k^*| - \ln|\bm{\xi}_k|\right]
\end{equation}

\textbf{Key advantages}:
\begin{itemize}
    \item Source positions $\bm{\xi}_k$ appear continuously---no mesh discretization required
    \item Exact evaluation at any boundary point
    \item Analytical gradients available for optimization
\end{itemize}


% ============================================================================
% SECTION 4: BOUNDARY ELEMENT METHOD
% ============================================================================
\section{Boundary Element Method}
\label{sec:bem}

The Boundary Element Method (BEM) provides an alternative approach using numerical integration of the boundary integral equation \citep{sauter2011boundary,steinbach2008numerical}.

\subsection{Fundamental Solution}

The fundamental solution for the 2D Laplacian is:
\begin{equation}
    \label{eq:fundamental}
    \Phi(\vect{x}, \bm{\xi}) = -\frac{1}{2\pi} \ln|\vect{x} - \bm{\xi}|
\end{equation}

\subsection{BEM Discretization}

We discretize the boundary into $n_e$ elements with collocation points at element midpoints. For point sources, the BEM evaluates:
\begin{equation}
    \label{eq:bem_forward}
    u(\vect{x}_i) = \sum_{k=1}^{N} q_k \, G_{\text{BEM}}(\vect{x}_i, \bm{\xi}_k)
\end{equation}
where $G_{\text{BEM}}$ is computed via numerical integration using Gaussian quadrature.

\subsection{Comparison with Analytical Solution}

For the unit disk, the BEM and analytical solutions should agree:
\begin{equation}
    G_{\text{BEM}}(\vect{x}, \bm{\xi}) \approx G_D(\vect{x}, \bm{\xi})
\end{equation}
with differences due only to quadrature error. This provides a valuable validation tool.


% ============================================================================
% SECTION 5: FINITE ELEMENT METHOD
% ============================================================================
\section{Finite Element Method}
\label{sec:fem}

The Finite Element Method (FEM) provides a systematic approach to discretizing the weak formulation on general domains \citep{brenner2008mathematical,ern2004theory}.

\subsection{Discrete Problem}

Using piecewise linear (P1) Lagrange elements on a triangular mesh $\mathcal{T}_h$:
\begin{equation}
    \label{eq:fem_system}
    \mat{A} \vect{u} = \vect{b}
\end{equation}
where $A_{ij} = \int_\Omega \grad \phi_j \cdot \grad \phi_i \, \dd\vect{x}$ and $b_i = \sum_{k=1}^{N} q_k \, \phi_i(\bm{\xi}_k)$.

\subsection{Load Vector Assembly}

The load vector requires evaluating basis functions at source positions using barycentric coordinates:
\begin{equation}
    \label{eq:interp_method}
    b_i = \sum_{k=1}^{N} q_k \, \lambda_i(\bm{\xi}_k)
\end{equation}

\textbf{Key property}: This makes $\vect{b}$ a \emph{piecewise linear} function of source position, affecting the optimization landscape.


% ============================================================================
% SECTION 6: CONFORMAL MAPPING
% ============================================================================
\section{Conformal Mapping for General Domains}
\label{sec:conformal}

Conformal mapping techniques extend the analytical approach to general simply connected domains \citep{ablowitz2003complex,driscoll2002schwarz}.

\begin{theorem}[Conformal Invariance]
\label{thm:conformal_invariance}
The Green's functions of conformally equivalent domains are related by:
\begin{equation}
    \label{eq:greens_conformal}
    G_\Omega(\vect{z}_1, \vect{z}_2) = G_D(f(\vect{z}_1), f(\vect{z}_2))
\end{equation}
where $f: \Omega \to D$ is the conformal map.
\end{theorem}

This means for any simply connected domain $\Omega$:
\begin{enumerate}
    \item Map source and boundary points to the disk via $f$
    \item Use the analytical disk Green's function \eqref{eq:disk_greens}
    \item The result is automatically correct for the physical domain
\end{enumerate}


% ============================================================================
% SECTION 7: INVERSE PROBLEM
% ============================================================================
\section{Inverse Problem Formulations}
\label{sec:inverse}

\subsection{Problem Statement}

Given boundary measurements $u^{\text{meas}}$, find source parameters $\{(\bm{\xi}_k, q_k)\}$ such that $u(\vect{x}) \approx u^{\text{meas}}(\vect{x})$ on $\boundary$, subject to $\sum_k q_k = 0$.

\subsection{Ill-Posedness}

The inverse source problem is ill-posed \citep{hadamard1923lectures,kirsch2011introduction}: non-unique and unstable. Regularization is essential.

\subsection{Two Formulations}

\textbf{Nonlinear}: Optimize positions and intensities continuously:
\begin{equation}
    \min_{\{\bm{\xi}_k, q_k\}} \norm{u(\cdot; \bm{\xi}, \vect{q}) - u^{\text{meas}}}^2
\end{equation}

\textbf{Linear}: Fix positions on a grid, solve for intensities:
\begin{equation}
    \min_{\vect{q}} \norm{\mat{G}\vect{q} - \vect{u}^{\text{meas}}}^2 + \mathcal{R}(\vect{q})
\end{equation}


% ============================================================================
% SECTION 8: REGULARIZATION
% ============================================================================
\section{Regularization Methods}
\label{sec:regularization}

\subsection{Tikhonov ($L^2$)}
\begin{equation}
    \min_{\vect{q}} \norm{\mat{G}\vect{q} - \vect{u}^{\text{meas}}}_2^2 + \alpha \norm{\vect{q}}_2^2
\end{equation}
Closed-form solution; produces smooth (diffuse) reconstructions.

\subsection{Sparsity ($L^1$)}
\begin{equation}
    \min_{\vect{q}} \norm{\mat{G}\vect{q} - \vect{u}^{\text{meas}}}_2^2 + \alpha \norm{\vect{q}}_1
\end{equation}
Promotes sparse solutions \citep{tibshirani1996regression}; better for point sources.

\subsection{Total Variation}
\begin{equation}
    \min_{\vect{q}} \norm{\mat{G}\vect{q} - \vect{u}^{\text{meas}}}_2^2 + \alpha \norm{\mat{D}\vect{q}}_1
\end{equation}
Promotes piecewise constant solutions \citep{rudin1992nonlinear}; solved via ADMM \citep{boyd2011distributed} or Chambolle-Pock \citep{chambolle2011first}.

\textbf{Key insight}: For point sources, $L^1$ outperforms TV because it promotes sparse \emph{values} rather than sparse \emph{gradients}.


% ============================================================================
% SECTION 9: NUMERICAL EXPERIMENTS
% ============================================================================
\section{Numerical Experiments}
\label{sec:numerical}

\subsection{Test Problem}

Four point sources in the unit disk with positions $(-0.3, 0.4)$, $(0.5, 0.3)$, $(-0.4, -0.4)$, $(0.3, -0.5)$ and intensities $\pm 1$.

\subsection{Results Summary}

\begin{table}[H]
\centering
\caption{Solver comparison (Position RMSE / Time)}
\label{tab:results_summary}
\begin{tabular}{lccc}
\toprule
& \textbf{Analytical} & \textbf{BEM} & \textbf{FEM} \\
\midrule
Linear $L^1$ & 0.259 / 0.1s & 0.273 / 0.1s & 0.273 / 0.5s \\
Linear $L^2$ & 0.052 / 0.0s & 0.052 / 0.2s & 0.052 / 0.5s \\
Nonlinear (DE) & 0.235 / 4.9s & 0.269 / 31s & \textbf{0.004} / 58s \\
\bottomrule
\end{tabular}
\end{table}

\subsection{Key Finding: Optimization Landscape}

The FEM nonlinear solver achieved near-perfect recovery (RMSE = 0.004), far outperforming analytical and BEM versions. This is due to FEM's \textbf{smoother optimization landscape}:
\begin{itemize}
    \item FEM's piecewise linear interpolation creates broader valleys
    \item The mesh discretization limits sharp gradients
    \item Fewer narrow local minima trap the optimizer
\end{itemize}

Analytical and BEM have equivalent landscapes (same Green's function), so their performance differences are due to random variation in restarts.


% ============================================================================
% SECTION 10: IMPLEMENTATION
% ============================================================================
\section{Implementation}
\label{sec:implementation}

\subsection{Software Modules}

\begin{itemize}
    \item \texttt{analytical\_solver.py}: Exact Green's function
    \item \texttt{bem\_solver.py}: Numerical BEM
    \item \texttt{fem\_solver.py}: FEM via scikit-fem \citep{gustafsson2020scikit}
    \item \texttt{conformal\_solver.py}: Conformal mapping
    \item \texttt{regularization.py}: $L^1$, $L^2$, TV
    \item \texttt{comparison.py}: Comprehensive benchmarks
\end{itemize}

\subsection{Dependencies}

NumPy \citep{harris2020array}, SciPy \citep{virtanen2020scipy}, scikit-fem \citep{gustafsson2020scikit}, Matplotlib.


% ============================================================================
% SECTION 11: CONCLUSIONS
% ============================================================================
\section{Conclusions}
\label{sec:conclusions}

This report presented mathematical foundations and numerical methods for inverse source localization. Key findings:

\begin{enumerate}
    \item Three forward solvers (Analytical, BEM, FEM) produce equivalent results for linear inverse problems
    \item For nonlinear problems, FEM's smoother landscape enables better global optimization
    \item $L^1$ regularization outperforms TV for point source recovery
    \item Conformal mapping extends analytical methods to general domains
\end{enumerate}


% ============================================================================
% BIBLIOGRAPHY
% ============================================================================
\newpage
\bibliographystyle{abbrvnat}
\bibliography{references}

\end{document}
