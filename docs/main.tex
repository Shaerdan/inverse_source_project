\documentclass[11pt,a4paper]{article}

% ============================================================================
% PACKAGES
% ============================================================================
\usepackage[utf8]{inputenc}
\usepackage[T1]{fontenc}
\usepackage{amsmath,amssymb,amsthm}
\usepackage{mathtools}
\usepackage{bm}
\usepackage{geometry}
\usepackage{graphicx}
\usepackage{float}
\usepackage{booktabs}
\usepackage{array}
\usepackage{algorithm}
\usepackage{algpseudocode}
\usepackage{listings}
\usepackage{xcolor}
\usepackage{hyperref}
\usepackage{cleveref}
\usepackage{natbib}
\usepackage{caption}
\usepackage{subcaption}
\usepackage{tikz}
\usetikzlibrary{shapes,arrows,positioning}

% ============================================================================
% PAGE SETUP
% ============================================================================
\geometry{margin=1in}
\setlength{\parindent}{0pt}
\setlength{\parskip}{6pt}

% ============================================================================
% THEOREM ENVIRONMENTS
% ============================================================================
\theoremstyle{plain}
\newtheorem{theorem}{Theorem}[section]
\newtheorem{lemma}[theorem]{Lemma}
\newtheorem{proposition}[theorem]{Proposition}
\newtheorem{corollary}[theorem]{Corollary}

\theoremstyle{definition}
\newtheorem{definition}[theorem]{Definition}
\newtheorem{example}[theorem]{Example}
\newtheorem{remark}[theorem]{Remark}
\newtheorem{problem}{Problem}[section]

% ============================================================================
% CUSTOM COMMANDS
% ============================================================================
\newcommand{\R}{\mathbb{R}}
\newcommand{\C}{\mathbb{C}}
\newcommand{\N}{\mathbb{N}}
\newcommand{\boundary}{\partial\Omega}
\newcommand{\inner}[2]{\langle #1, #2 \rangle}
\newcommand{\norm}[1]{\left\| #1 \right\|}
\newcommand{\abs}[1]{\left| #1 \right|}
\newcommand{\dd}{\mathrm{d}}
\newcommand{\pp}{\partial}
\newcommand{\grad}{\nabla}
\newcommand{\divergence}{\nabla \cdot}
\newcommand{\laplacian}{\Delta}
\newcommand{\vect}[1]{\mathbf{#1}}
\newcommand{\mat}[1]{\mathbf{#1}}
\newcommand{\dirac}{\delta}
\newcommand{\argmin}{\operatorname{argmin}}

% Code listing style
\lstset{
    basicstyle=\small\ttfamily,
    keywordstyle=\color{blue},
    commentstyle=\color{gray},
    stringstyle=\color{red},
    numbers=left,
    numberstyle=\tiny,
    numbersep=5pt,
    frame=single,
    breaklines=true
}

% ============================================================================
% DOCUMENT
% ============================================================================
\begin{document}

% ============================================================================
% TITLE PAGE
% ============================================================================
\begin{titlepage}
    \centering
    \vspace*{2cm}
    
    {\Huge\bfseries Inverse Source Localization\\[0.5cm] for the Poisson Equation}
    
    \vspace{1cm}
    
    {\Large Mathematical Foundations and Numerical Methods}
    
    \vspace{2cm}
    
    {\large Technical Report}
    
    \vspace{1cm}
    
    {\large\today}
    
    \vfill
    
    \begin{abstract}
    \noindent
    This report presents a comprehensive mathematical framework for inverse source localization problems governed by the Poisson equation. We develop both the forward problem—computing boundary measurements from interior point sources—and the inverse problem—recovering source locations and intensities from boundary data. Two complementary numerical approaches are presented: the Finite Element Method (FEM) for general domains and the Boundary Element Method (BEM) with conformal mapping for simply connected domains. For the inverse problem, we formulate both nonlinear optimization (for continuous source positions) and linear algebraic approaches (for discretized source grids) with various regularization strategies including Tikhonov ($L^2$), sparsity-promoting ($L^1$), and Total Variation (TV). Complete algorithmic details and implementation considerations are provided.
    \end{abstract}
    
\end{titlepage}

% ============================================================================
% TABLE OF CONTENTS
% ============================================================================
\tableofcontents
\newpage

% ============================================================================
% SECTION 1: INTRODUCTION
% ============================================================================
\section{Introduction}
\label{sec:introduction}

The inverse source problem for elliptic partial differential equations arises in numerous applications including geophysical prospecting \citep{isakov2006inverse}, medical imaging \citep{ammari2004reconstruction}, environmental monitoring \citep{el2005identification}, and non-destructive testing \citep{andrieux2006inverse}. The fundamental challenge is to determine the location and strength of interior sources from measurements taken only on the boundary of the domain.

\subsection{Problem Overview}

Consider a bounded domain $\Omega \subset \R^2$ with boundary $\boundary$. The forward problem consists of solving the Poisson equation with point sources:
\begin{equation}
    \label{eq:forward_intro}
    -\laplacian u = \sum_{k=1}^{N} q_k \dirac(\vect{x} - \bm{\xi}_k) \quad \text{in } \Omega
\end{equation}
subject to appropriate boundary conditions, where $\bm{\xi}_k \in \Omega$ are source locations and $q_k \in \R$ are source intensities.

The inverse problem seeks to recover the source parameters $\{(\bm{\xi}_k, q_k)\}_{k=1}^{N}$ from boundary measurements of $u$ or its derivatives.

\subsection{Scope and Organization}

This report is organized as follows:
\begin{itemize}
    \item \textbf{Section~\ref{sec:forward}}: Mathematical formulation of the forward problem
    \item \textbf{Section~\ref{sec:fem}}: Finite Element Method discretization
    \item \textbf{Section~\ref{sec:bem}}: Boundary Element Method with Green's functions
    \item \textbf{Section~\ref{sec:conformal}}: Conformal mapping for general domains
    \item \textbf{Section~\ref{sec:inverse}}: Inverse problem formulations
    \item \textbf{Section~\ref{sec:regularization}}: Regularization methods
    \item \textbf{Section~\ref{sec:algorithms}}: Numerical algorithms
    \item \textbf{Section~\ref{sec:implementation}}: Implementation details
\end{itemize}


% ============================================================================
% SECTION 2: FORWARD PROBLEM
% ============================================================================
\section{The Forward Problem}
\label{sec:forward}

\subsection{Strong Formulation}

Let $\Omega \subset \R^2$ be a bounded, simply connected domain with smooth boundary $\boundary$. We consider the Poisson equation with point sources and homogeneous Neumann boundary conditions:

\begin{problem}[Strong Form]
\label{prob:strong}
Find $u: \Omega \to \R$ such that
\begin{align}
    -\laplacian u &= f \quad \text{in } \Omega \label{eq:poisson}\\
    \frac{\pp u}{\pp \vect{n}} &= 0 \quad \text{on } \boundary \label{eq:neumann}
\end{align}
where $\vect{n}$ is the outward unit normal to $\boundary$, and the source term is
\begin{equation}
    \label{eq:source_term}
    f(\vect{x}) = \sum_{k=1}^{N} q_k \dirac(\vect{x} - \bm{\xi}_k)
\end{equation}
with source locations $\bm{\xi}_k \in \Omega$ and intensities $q_k \in \R$.
\end{problem}

\subsection{Compatibility Condition}

For the Neumann problem to admit a solution, the source term must satisfy a compatibility condition. Integrating \eqref{eq:poisson} over $\Omega$ and applying the divergence theorem:
\begin{equation}
    -\int_\Omega \laplacian u \, \dd\vect{x} = -\int_{\boundary} \frac{\pp u}{\pp \vect{n}} \, \dd s = 0
\end{equation}

Thus, we require:
\begin{equation}
    \label{eq:compatibility}
    \int_\Omega f \, \dd\vect{x} = \sum_{k=1}^{N} q_k = 0
\end{equation}

\begin{remark}[Physical Interpretation]
The compatibility condition \eqref{eq:compatibility} states that the total source strength must equal the total sink strength. This reflects conservation: with no flux through the boundary, what flows out of sources must flow into sinks.
\end{remark}

\subsection{Uniqueness}

The solution to Problem~\ref{prob:strong} is unique only up to an additive constant. We fix this ambiguity by imposing:
\begin{equation}
    \label{eq:zero_mean}
    \int_\Omega u \, \dd\vect{x} = 0
\end{equation}

\subsection{Weak Formulation}

The presence of Dirac delta distributions in \eqref{eq:source_term} requires a weak (variational) formulation. Let $H^1(\Omega)$ denote the Sobolev space of functions with square-integrable weak derivatives.

Multiplying \eqref{eq:poisson} by a test function $v \in H^1(\Omega)$ and integrating by parts:
\begin{align}
    \int_\Omega (-\laplacian u) v \, \dd\vect{x} &= \int_\Omega f v \, \dd\vect{x} \\
    \int_\Omega \grad u \cdot \grad v \, \dd\vect{x} - \int_{\boundary} \frac{\pp u}{\pp \vect{n}} v \, \dd s &= \int_\Omega f v \, \dd\vect{x}
\end{align}

With the Neumann condition \eqref{eq:neumann}, the boundary integral vanishes.

\begin{problem}[Weak Form]
\label{prob:weak}
Find $u \in H^1(\Omega)$ with $\int_\Omega u \, \dd\vect{x} = 0$ such that for all $v \in H^1(\Omega)$:
\begin{equation}
    \label{eq:weak_form}
    \int_\Omega \grad u \cdot \grad v \, \dd\vect{x} = \sum_{k=1}^{N} q_k \, v(\bm{\xi}_k)
\end{equation}
\end{problem}

\begin{remark}[Point Evaluation]
The right-hand side of \eqref{eq:weak_form} involves point evaluation $v(\bm{\xi}_k)$, which is well-defined for $v \in H^1(\Omega)$ in two dimensions by the Sobolev embedding theorem \citep{evans2010partial}.
\end{remark}

\subsection{Green's Function Representation}

The solution to Problem~\ref{prob:strong} can be expressed using the Neumann Green's function.

\begin{definition}[Neumann Green's Function]
\label{def:greens}
The Neumann Green's function $G: \Omega \times \Omega \to \R$ satisfies:
\begin{align}
    -\laplacian_{\vect{x}} G(\vect{x}, \bm{\xi}) &= \dirac(\vect{x} - \bm{\xi}) - \frac{1}{|\Omega|} \quad \text{in } \Omega \label{eq:greens_pde}\\
    \frac{\pp G}{\pp \vect{n}_{\vect{x}}} &= 0 \quad \text{on } \boundary \label{eq:greens_bc}\\
    \int_\Omega G(\vect{x}, \bm{\xi}) \, \dd\vect{x} &= 0 \label{eq:greens_norm}
\end{align}
\end{definition}

The term $-1/|\Omega|$ in \eqref{eq:greens_pde} ensures compatibility with the Neumann condition. The solution to Problem~\ref{prob:strong} is then:
\begin{equation}
    \label{eq:greens_representation}
    u(\vect{x}) = \sum_{k=1}^{N} q_k \, G(\vect{x}, \bm{\xi}_k)
\end{equation}

\begin{theorem}[Properties of $G$]
\label{thm:greens_properties}
The Neumann Green's function satisfies:
\begin{enumerate}
    \item \textbf{Symmetry}: $G(\vect{x}, \bm{\xi}) = G(\bm{\xi}, \vect{x})$
    \item \textbf{Singularity}: $G(\vect{x}, \bm{\xi}) = -\frac{1}{2\pi}\ln|\vect{x} - \bm{\xi}| + H(\vect{x}, \bm{\xi})$ where $H$ is smooth
    \item \textbf{Continuity in $\bm{\xi}$}: $G(\vect{x}, \cdot)$ is continuous for $\vect{x} \neq \bm{\xi}$
\end{enumerate}
\end{theorem}

The continuity property is crucial for inverse problems: it ensures that boundary measurements vary smoothly as source positions change.


% ============================================================================
% SECTION 3: FINITE ELEMENT METHOD
% ============================================================================
\section{Finite Element Method}
\label{sec:fem}

The Finite Element Method (FEM) provides a systematic approach to discretizing the weak formulation \eqref{eq:weak_form} on general domains \citep{brenner2008mathematical,ern2004theory}.

\subsection{Triangular Mesh}

Let $\mathcal{T}_h$ be a conforming triangulation of $\Omega$ with mesh parameter $h > 0$ representing the maximum element diameter. The mesh consists of:
\begin{itemize}
    \item Nodes: $\{\vect{x}_1, \vect{x}_2, \ldots, \vect{x}_n\}$
    \item Triangular elements: $\{T_1, T_2, \ldots, T_m\}$
\end{itemize}

\subsection{Finite Element Space}

We use piecewise linear (P1) Lagrange finite elements:
\begin{equation}
    \label{eq:fe_space}
    V_h = \{v_h \in C^0(\bar{\Omega}) : v_h|_T \in \mathcal{P}_1(T) \text{ for all } T \in \mathcal{T}_h\}
\end{equation}
where $\mathcal{P}_1(T)$ denotes polynomials of degree at most 1 on triangle $T$.

\subsection{Basis Functions}

The space $V_h$ has dimension $n$ (number of nodes) with nodal basis functions $\{\phi_1, \phi_2, \ldots, \phi_n\}$ satisfying:
\begin{equation}
    \label{eq:nodal_basis}
    \phi_i(\vect{x}_j) = \delta_{ij} = \begin{cases} 1 & \text{if } i = j \\ 0 & \text{if } i \neq j \end{cases}
\end{equation}

Each $\phi_i$ is supported only on triangles containing node $\vect{x}_i$ and is linear on each triangle.

\subsection{Barycentric Coordinates}

For a triangle $T$ with vertices $\vect{x}_1^T, \vect{x}_2^T, \vect{x}_3^T$, any point $\vect{x} \in T$ can be written as:
\begin{equation}
    \vect{x} = \lambda_1 \vect{x}_1^T + \lambda_2 \vect{x}_2^T + \lambda_3 \vect{x}_3^T
\end{equation}
where $\lambda_1 + \lambda_2 + \lambda_3 = 1$ and $\lambda_i \geq 0$ are the \emph{barycentric coordinates}.

Explicitly, for $\vect{x} = (x, y)$ and $\vect{x}_i^T = (x_i, y_i)$:
\begin{align}
    \lambda_1 &= \frac{(y_2 - y_3)(x - x_3) + (x_3 - x_2)(y - y_3)}{\det J} \\
    \lambda_2 &= \frac{(y_3 - y_1)(x - x_3) + (x_1 - x_3)(y - y_3)}{\det J} \\
    \lambda_3 &= 1 - \lambda_1 - \lambda_2
\end{align}
where $\det J = (y_2 - y_3)(x_1 - x_3) + (x_3 - x_2)(y_1 - y_3)$ is twice the signed area of $T$.

The restriction of basis function $\phi_i$ to triangle $T$ containing node $i$ equals the corresponding barycentric coordinate:
\begin{equation}
    \label{eq:basis_barycentric}
    \phi_i|_T = \lambda_i
\end{equation}

\subsection{Discrete Problem}

Approximating $u \approx u_h = \sum_{j=1}^{n} u_j \phi_j$, the weak form \eqref{eq:weak_form} becomes:
\begin{equation}
    \sum_{j=1}^{n} u_j \int_\Omega \grad \phi_j \cdot \grad \phi_i \, \dd\vect{x} = \sum_{k=1}^{N} q_k \, \phi_i(\bm{\xi}_k) \quad \text{for } i = 1, \ldots, n
\end{equation}

In matrix form:
\begin{equation}
    \label{eq:fem_system}
    \mat{A} \vect{u} = \vect{b}
\end{equation}
where:
\begin{align}
    A_{ij} &= \int_\Omega \grad \phi_j \cdot \grad \phi_i \, \dd\vect{x} \quad \text{(stiffness matrix)} \label{eq:stiffness}\\
    b_i &= \sum_{k=1}^{N} q_k \, \phi_i(\bm{\xi}_k) \quad \text{(load vector)} \label{eq:load_vector}
\end{align}

\subsection{Stiffness Matrix Properties}

\begin{proposition}
The stiffness matrix $\mat{A}$ satisfies:
\begin{enumerate}
    \item $\mat{A}$ is symmetric positive semi-definite
    \item $\mat{A}$ has a one-dimensional null space: $\mat{A} \vect{1} = \vect{0}$ where $\vect{1} = (1, \ldots, 1)^\top$
    \item $\mat{A}$ is sparse with bandwidth $\mathcal{O}(1)$ per row
\end{enumerate}
\end{proposition}

\subsection{Load Vector Assembly: Two Methods}
\label{sec:fem_load_vector}

The load vector \eqref{eq:load_vector} requires evaluating $\phi_i(\bm{\xi}_k)$ for source positions $\bm{\xi}_k$ that may not coincide with mesh nodes.

\subsubsection{Method A: Nodal Snapping (Approximate)}

The simplest approach is to assign each source to its nearest mesh node:
\begin{equation}
    \label{eq:snap_method}
    b_i = \sum_{k: i = i^*_k} q_k, \quad \text{where } i^*_k = \argmin_{j} \|\vect{x}_j - \bm{\xi}_k\|
\end{equation}

\textbf{Advantages}: Simple implementation, fast (no geometric search).

\textbf{Disadvantages}: 
\begin{itemize}
    \item Source location error up to $h/2$
    \item Discontinuous dependence on $\bm{\xi}_k$: as $\bm{\xi}_k$ crosses Voronoi cell boundaries, the effective source jumps to a different node
    \item Objective function for inverse problems becomes piecewise constant
\end{itemize}

\subsubsection{Method B: Barycentric Interpolation (Exact)}

The mathematically correct approach uses \eqref{eq:basis_barycentric}:
\begin{equation}
    \label{eq:interp_method}
    b_i = \sum_{k=1}^{N} q_k \, \lambda_i(\bm{\xi}_k)
\end{equation}
where $\lambda_i(\bm{\xi}_k)$ is the barycentric coordinate of $\bm{\xi}_k$ with respect to node $i$ in the containing triangle.

\textbf{Algorithm}:
\begin{enumerate}
    \item Find triangle $T_k$ containing $\bm{\xi}_k$ (point location)
    \item Compute barycentric coordinates $(\lambda_1, \lambda_2, \lambda_3)$
    \item Add contributions: $b_{i_1} \mathrel{+}= q_k \lambda_1$, $b_{i_2} \mathrel{+}= q_k \lambda_2$, $b_{i_3} \mathrel{+}= q_k \lambda_3$
\end{enumerate}

\textbf{Advantages}:
\begin{itemize}
    \item Exact Galerkin discretization of weak form
    \item Continuous dependence on $\bm{\xi}_k$
    \item Smooth objective function for gradient-based optimization
\end{itemize}

\subsection{Handling the Null Space}

Since $\mat{A}$ is singular, the system \eqref{eq:fem_system} requires special treatment:

\begin{proposition}[Solvability]
The system $\mat{A}\vect{u} = \vect{b}$ has a solution if and only if $\vect{b} \perp \ker(\mat{A})$, i.e., $\sum_i b_i = 0$.
\end{proposition}

For our load vector: $\sum_i b_i = \sum_i \sum_k q_k \phi_i(\bm{\xi}_k) = \sum_k q_k \sum_i \phi_i(\bm{\xi}_k) = \sum_k q_k \cdot 1 = \sum_k q_k$

Thus, solvability requires $\sum_k q_k = 0$, which is precisely the compatibility condition \eqref{eq:compatibility}.

\textbf{Practical Solution}: Use a sparse direct solver and project to zero mean:
\begin{equation}
    \vect{u} \leftarrow \vect{u} - \text{mean}(\vect{u}) \cdot \vect{1}
\end{equation}


% ============================================================================
% SECTION 4: BOUNDARY ELEMENT METHOD
% ============================================================================
\section{Boundary Element Method}
\label{sec:bem}

The Boundary Element Method (BEM) offers an alternative approach that requires discretization only on the boundary $\boundary$, not the interior \citep{sauter2011boundary,steinbach2008numerical}.

\subsection{Fundamental Solution}

The fundamental solution (free-space Green's function) for the 2D Laplacian is:
\begin{equation}
    \label{eq:fundamental}
    \Phi(\vect{x}, \bm{\xi}) = -\frac{1}{2\pi} \ln|\vect{x} - \bm{\xi}|
\end{equation}
satisfying $-\laplacian_{\vect{x}} \Phi(\vect{x}, \bm{\xi}) = \dirac(\vect{x} - \bm{\xi})$ in $\R^2$.

\subsection{Green's Function Decomposition}

The domain Green's function decomposes as:
\begin{equation}
    \label{eq:greens_decomposition}
    G(\vect{x}, \bm{\xi}) = \Phi(\vect{x}, \bm{\xi}) + H(\vect{x}, \bm{\xi})
\end{equation}
where $H$ is the \emph{regular part} satisfying:
\begin{align}
    \laplacian_{\vect{x}} H(\vect{x}, \bm{\xi}) &= \frac{1}{|\Omega|} \quad \text{in } \Omega \\
    \frac{\pp H}{\pp \vect{n}} &= -\frac{\pp \Phi}{\pp \vect{n}} \quad \text{on } \boundary
\end{align}

\subsection{Unit Disk: Analytical Green's Function}

For the unit disk $D = \{\vect{x} \in \R^2 : |\vect{x}| < 1\}$, the Neumann Green's function has an explicit form using the method of images.

\begin{theorem}[Green's Function for Unit Disk]
\label{thm:disk_greens}
For the unit disk with Neumann boundary conditions:
\begin{equation}
    \label{eq:disk_greens}
    G_D(\vect{x}, \bm{\xi}) = -\frac{1}{2\pi}\left[\ln|\vect{x} - \bm{\xi}| + \ln|\vect{x} - \bm{\xi}^*| - \ln|\bm{\xi}|\right] + C
\end{equation}
where $\bm{\xi}^* = \bm{\xi}/|\bm{\xi}|^2$ is the image point (Kelvin transform) and $C$ is chosen to satisfy the normalization \eqref{eq:greens_norm}.
\end{theorem}

\begin{proof}[Sketch]
The image point $\bm{\xi}^*$ lies outside $D$ and is positioned such that for $\vect{x} \in \pp D$:
\begin{equation}
    \frac{\pp}{\pp \vect{n}}\left[\ln|\vect{x} - \bm{\xi}| + \ln|\vect{x} - \bm{\xi}^*|\right] = 0
\end{equation}
This follows from the reflection property of the Kelvin transform.
\end{proof}

\subsection{BEM Forward Solver}

Using \eqref{eq:greens_representation}, the solution at any point $\vect{x} \in \Omega$ is:
\begin{equation}
    \label{eq:bem_forward}
    u(\vect{x}) = \sum_{k=1}^{N} q_k \, G(\vect{x}, \bm{\xi}_k)
\end{equation}

\textbf{Key Advantage}: Source positions $\bm{\xi}_k$ appear continuously in \eqref{eq:bem_forward}—no mesh discretization of the interior is required.

For boundary measurements, we evaluate \eqref{eq:bem_forward} at boundary points $\vect{x} \in \boundary$:
\begin{equation}
    \label{eq:boundary_measurement}
    u(\vect{x}) = \sum_{k=1}^{N} q_k \, G(\vect{x}, \bm{\xi}_k), \quad \vect{x} \in \boundary
\end{equation}


% ============================================================================
% SECTION 5: CONFORMAL MAPPING
% ============================================================================
\section{Conformal Mapping for General Domains}
\label{sec:conformal}

A key advantage of working in two dimensions is the availability of conformal mapping techniques, which allow us to transform problems on general domains to the unit disk where analytical solutions exist \citep{ablowitz2003complex,driscoll2002schwarz}.

\subsection{Conformal Invariance of the Laplacian}

\begin{theorem}[Conformal Invariance]
\label{thm:conformal_invariance}
Let $f: \Omega \to D$ be a conformal (angle-preserving, analytic) map from domain $\Omega$ to the unit disk $D$. If $u$ is harmonic in $\Omega$, then $\tilde{u} = u \circ f^{-1}$ is harmonic in $D$.
\end{theorem}

\begin{corollary}[Green's Function Transformation]
\label{cor:greens_transform}
The Green's functions of conformally equivalent domains are related by:
\begin{equation}
    \label{eq:greens_conformal}
    G_\Omega(\vect{z}_1, \vect{z}_2) = G_D(f(\vect{z}_1), f(\vect{z}_2))
\end{equation}
where $f: \Omega \to D$ is the conformal map.
\end{corollary}

This remarkable result means that for any simply connected domain $\Omega$, we can:
\begin{enumerate}
    \item Compute the conformal map $f: \Omega \to D$
    \item Use the analytical disk Green's function \eqref{eq:disk_greens}
    \item Maintain truly continuous source positions (no mesh!)
\end{enumerate}

\subsection{Riemann Mapping Theorem}

\begin{theorem}[Riemann Mapping Theorem]
\label{thm:riemann}
Let $\Omega \subset \C$ be a simply connected domain with $\Omega \neq \C$. Then there exists a unique conformal map $f: \Omega \to D$ satisfying $f(z_0) = 0$ and $f'(z_0) > 0$ for a specified interior point $z_0 \in \Omega$.
\end{theorem}

\subsection{Conformal Maps for Specific Domains}

\subsubsection{Ellipse}

For an ellipse with semi-axes $a > b$, the conformal map from the unit disk is given by a Joukowsky-type transformation:
\begin{equation}
    \label{eq:ellipse_map}
    z = \frac{a+b}{2} w + \frac{a-b}{2} \frac{1}{w}
\end{equation}
where $w \in D$ and $z \in \Omega$ (ellipse interior).

The inverse map (from ellipse to disk) is:
\begin{equation}
    \label{eq:ellipse_inverse}
    w = \frac{z - \sqrt{z^2 - (a^2 - b^2)}}{a + b}
\end{equation}
choosing the branch with $|w| < 1$.

\subsubsection{Star-Shaped Domains}

For a star-shaped domain with boundary $r = r(\theta)$ in polar coordinates, numerical conformal mapping methods are required. The Theodorsen integral equation or iterative methods can compute the boundary correspondence \citep{henrici1986applied}.

\subsubsection{Polygons: Schwarz-Christoffel Mapping}

For polygonal domains, the Schwarz-Christoffel formula provides the conformal map:
\begin{equation}
    \label{eq:schwarz_christoffel}
    f(z) = A \int^z \prod_{k=1}^{n} (\zeta - z_k)^{-\beta_k} \, d\zeta + B
\end{equation}
where $z_k$ are prevertices on the unit circle and $\beta_k = 1 - \alpha_k/\pi$ with $\alpha_k$ the interior angles.

\begin{remark}[Singularities at Corners]
The Schwarz-Christoffel map has singularities at polygon vertices, which can cause numerical difficulties. Specialized algorithms and software (e.g., SC Toolbox) are recommended \citep{driscoll2002schwarz}.
\end{remark}

\subsection{Conformal BEM Algorithm}

\begin{algorithm}[H]
\caption{Conformal BEM Forward Solver}
\label{alg:conformal_bem}
\begin{algorithmic}[1]
\Require Conformal map $f: \Omega \to D$, sources $\{(\bm{\xi}_k, q_k)\}$, boundary points $\{\vect{x}_i\}$
\Ensure Boundary values $\{u_i\}$
\For{$i = 1, \ldots, n_{\text{boundary}}$}
    \State $u_i \gets 0$
    \For{$k = 1, \ldots, N$}
        \State $w_x \gets f(\vect{x}_i)$ \Comment{Map boundary point to disk}
        \State $w_\xi \gets f(\bm{\xi}_k)$ \Comment{Map source to disk}
        \State $u_i \gets u_i + q_k \cdot G_D(w_x, w_\xi)$ \Comment{Use disk Green's function}
    \EndFor
\EndFor
\State $u_i \gets u_i - \text{mean}(\{u_i\})$ \Comment{Zero mean normalization}
\end{algorithmic}
\end{algorithm}


% ============================================================================
% SECTION 6: INVERSE PROBLEM
% ============================================================================
\section{Inverse Problem Formulations}
\label{sec:inverse}

The inverse source problem seeks to recover source parameters from boundary measurements.

\subsection{Problem Statement}

\begin{problem}[Inverse Source Problem]
\label{prob:inverse}
Given boundary measurements $u^{\text{meas}}$ on $\boundary$, find source locations $\{\bm{\xi}_k\}_{k=1}^{N}$ and intensities $\{q_k\}_{k=1}^{N}$ such that the forward solution \eqref{eq:greens_representation} satisfies:
\begin{equation}
    u(\vect{x}) \approx u^{\text{meas}}(\vect{x}) \quad \text{for } \vect{x} \in \boundary
\end{equation}
subject to the compatibility constraint $\sum_k q_k = 0$.
\end{problem}

\subsection{Ill-Posedness}

The inverse source problem is \emph{ill-posed} in the sense of Hadamard \citep{hadamard1923lectures}:
\begin{itemize}
    \item \textbf{Non-uniqueness}: Multiple source configurations may produce identical boundary data
    \item \textbf{Instability}: Small perturbations in measurements can cause large changes in recovered sources
\end{itemize}

Regularization techniques (Section~\ref{sec:regularization}) are essential for obtaining stable, meaningful solutions.

\subsection{Two Formulations}

We present two complementary approaches:

\subsubsection{Formulation 1: Nonlinear Optimization (Continuous Source Positions)}

Treat both positions $\bm{\xi}_k$ and intensities $q_k$ as continuous unknowns:
\begin{equation}
    \label{eq:nonlinear_inverse}
    \min_{\{\bm{\xi}_k, q_k\}} \mathcal{J}(\bm{\xi}, \vect{q}) = \norm{u(\cdot; \bm{\xi}, \vect{q}) - u^{\text{meas}}}_{L^2(\boundary)}^2
\end{equation}
subject to $\bm{\xi}_k \in \Omega$ and $\sum_k q_k = 0$.

\textbf{Unknowns}: $3N - 1$ parameters (2 coordinates + 1 intensity per source, minus one intensity fixed by compatibility)

\textbf{Advantages}:
\begin{itemize}
    \item Source positions are truly continuous
    \item Low-dimensional optimization
    \item Physical interpretation of results
\end{itemize}

\textbf{Challenges}:
\begin{itemize}
    \item Non-convex optimization with local minima
    \item Requires number of sources $N$ to be specified
    \item Combinatorial complexity for matching recovered to true sources
\end{itemize}

\subsubsection{Formulation 2: Linear Inverse Problem (Discretized Source Grid)}

Fix source locations to a grid $\{\bm{\xi}_j\}_{j=1}^{M}$ and solve for intensities only:
\begin{equation}
    \label{eq:linear_inverse}
    \min_{\vect{q} \in \R^M} \norm{\mat{G}\vect{q} - \vect{u}^{\text{meas}}}^2 + \mathcal{R}(\vect{q})
\end{equation}
where $\mat{G}$ is the Green's matrix with entries:
\begin{equation}
    \label{eq:greens_matrix}
    G_{ij} = G(\vect{x}_i^{\text{boundary}}, \bm{\xi}_j^{\text{grid}})
\end{equation}
and $\mathcal{R}(\vect{q})$ is a regularization term.

\textbf{Unknowns}: $M$ intensity values at grid points

\textbf{Advantages}:
\begin{itemize}
    \item Linear (or convex) optimization
    \item Regularization theory well-developed
    \item Does not require specifying number of sources
\end{itemize}

\textbf{Challenges}:
\begin{itemize}
    \item Source positions constrained to grid
    \item High-dimensional (large $M$ for fine resolution)
    \item Intensity estimates may be diffuse
\end{itemize}

\subsection{Discretization of Objective Functionals}

For numerical implementation, we discretize boundary measurements at $n_b$ points $\{\vect{x}_i\}_{i=1}^{n_b}$:
\begin{equation}
    \mathcal{J} \approx \sum_{i=1}^{n_b} \left(u(\vect{x}_i) - u_i^{\text{meas}}\right)^2 = \norm{\mat{G}\vect{q} - \vect{u}^{\text{meas}}}_2^2
\end{equation}


% ============================================================================
% SECTION 7: REGULARIZATION
% ============================================================================
\section{Regularization Methods}
\label{sec:regularization}

Regularization is essential for stable inversion of ill-posed problems \citep{engl1996regularization,hansen2010discrete}.

\subsection{Tikhonov Regularization ($L^2$)}

The classical Tikhonov regularization penalizes the $L^2$ norm of the solution:
\begin{equation}
    \label{eq:tikhonov}
    \min_{\vect{q}} \norm{\mat{G}\vect{q} - \vect{u}^{\text{meas}}}_2^2 + \alpha \norm{\vect{q}}_2^2
\end{equation}

The solution is given by the normal equations:
\begin{equation}
    \label{eq:tikhonov_solution}
    \vect{q} = (\mat{G}^\top\mat{G} + \alpha \mat{I})^{-1} \mat{G}^\top \vect{u}^{\text{meas}}
\end{equation}

\textbf{Properties}:
\begin{itemize}
    \item Closed-form solution
    \item Smooth solutions (tends to smear out point sources)
    \item Well-understood regularization theory
\end{itemize}

\subsection{Sparsity Regularization ($L^1$)}

For point source recovery, $L^1$ regularization promotes sparse solutions:
\begin{equation}
    \label{eq:l1_regularization}
    \min_{\vect{q}} \norm{\mat{G}\vect{q} - \vect{u}^{\text{meas}}}_2^2 + \alpha \norm{\vect{q}}_1
\end{equation}
where $\norm{\vect{q}}_1 = \sum_j |q_j|$.

\textbf{IRLS Algorithm}: The Iteratively Reweighted Least Squares method approximates the $L^1$ problem:
\begin{equation}
    \label{eq:irls}
    \vect{q}^{(k+1)} = (\mat{G}^\top\mat{G} + \alpha \mat{W}^{(k)})^{-1} \mat{G}^\top \vect{u}^{\text{meas}}
\end{equation}
where $W_{jj}^{(k)} = 1/(|q_j^{(k)}| + \epsilon)$ with small $\epsilon > 0$.

\textbf{Properties}:
\begin{itemize}
    \item Promotes sparsity (many $q_j \approx 0$)
    \item Better localization of point sources than $L^2$
    \item Convex optimization problem
\end{itemize}

\subsection{Total Variation Regularization}

Total Variation (TV) regularization penalizes the total variation of the source distribution:
\begin{equation}
    \label{eq:tv_regularization}
    \min_{\vect{q}} \norm{\mat{G}\vect{q} - \vect{u}^{\text{meas}}}_2^2 + \alpha \text{TV}(\vect{q})
\end{equation}

For discrete domains, TV is defined via a gradient operator $\mat{D}$:
\begin{equation}
    \label{eq:discrete_tv}
    \text{TV}(\vect{q}) = \norm{\mat{D}\vect{q}}_1 = \sum_{\text{edges } (i,j)} |q_i - q_j|
\end{equation}

\textbf{Key Distinction from $L^1$}:
\begin{itemize}
    \item $L^1$: $\norm{\vect{q}}_1$ promotes sparse \emph{values}
    \item TV: $\norm{\mat{D}\vect{q}}_1$ promotes sparse \emph{gradients} (piecewise constant regions)
\end{itemize}

For point source recovery, $L^1$ is typically more appropriate than TV, as point sources have sparse values rather than sparse gradients.

\subsubsection{ADMM Algorithm for TV}

The Alternating Direction Method of Multipliers (ADMM) is effective for TV minimization \citep{boyd2011distributed}. We reformulate \eqref{eq:tv_regularization} as:
\begin{equation}
    \min_{\vect{q}, \vect{z}} \norm{\mat{G}\vect{q} - \vect{u}^{\text{meas}}}_2^2 + \alpha \norm{\vect{z}}_1 \quad \text{s.t. } \mat{D}\vect{q} = \vect{z}
\end{equation}

The augmented Lagrangian is:
\begin{equation}
    \mathcal{L}_\rho(\vect{q}, \vect{z}, \vect{w}) = \norm{\mat{G}\vect{q} - \vect{u}}_2^2 + \alpha\norm{\vect{z}}_1 + \frac{\rho}{2}\norm{\mat{D}\vect{q} - \vect{z} + \vect{w}}_2^2
\end{equation}

ADMM iterations:
\begin{align}
    \vect{q}^{(k+1)} &= (\mat{G}^\top\mat{G} + \rho\mat{D}^\top\mat{D})^{-1}(\mat{G}^\top\vect{u} + \rho\mat{D}^\top(\vect{z}^{(k)} - \vect{w}^{(k)})) \label{eq:admm_q}\\
    \vect{z}^{(k+1)} &= S_{\alpha/\rho}(\mat{D}\vect{q}^{(k+1)} + \vect{w}^{(k)}) \label{eq:admm_z}\\
    \vect{w}^{(k+1)} &= \vect{w}^{(k)} + \mat{D}\vect{q}^{(k+1)} - \vect{z}^{(k+1)} \label{eq:admm_w}
\end{align}
where $S_\tau(x) = \text{sign}(x) \max(|x| - \tau, 0)$ is the soft-thresholding operator.

\subsection{Parameter Selection: L-Curve Method}

The regularization parameter $\alpha$ balances data fidelity against regularization. The L-curve method \citep{hansen1992analysis} plots:
\begin{itemize}
    \item $x$-axis: $\log\norm{\mat{G}\vect{q}_\alpha - \vect{u}^{\text{meas}}}$ (residual)
    \item $y$-axis: $\log\norm{\vect{q}_\alpha}$ or $\log\mathcal{R}(\vect{q}_\alpha)$ (regularization norm)
\end{itemize}

The optimal $\alpha$ is located at the ``corner'' of the L-shaped curve, where curvature is maximized.


% ============================================================================
% SECTION 8: ALGORITHMS
% ============================================================================
\section{Numerical Algorithms}
\label{sec:algorithms}

\subsection{Nonlinear Optimization}

For the nonlinear inverse problem \eqref{eq:nonlinear_inverse}, we employ various optimization strategies.

\subsubsection{Global Optimizers}

Due to non-convexity, global optimization methods are often necessary:

\begin{itemize}
    \item \textbf{Differential Evolution} \citep{storn1997differential}: Population-based evolutionary algorithm; robust but slow
    \item \textbf{Basin Hopping}: Combines local optimization with random perturbations
    \item \textbf{Dual Annealing}: Simulated annealing variant with local search
\end{itemize}

\subsubsection{Gradient-Based Methods}

When a good initial guess is available, gradient-based methods are efficient:
\begin{itemize}
    \item \textbf{L-BFGS-B}: Limited-memory quasi-Newton with box constraints
    \item \textbf{Trust Region}: Newton-type with trust region constraints
\end{itemize}

The gradient of the objective \eqref{eq:nonlinear_inverse} with respect to source position $\bm{\xi}_k$ is:
\begin{equation}
    \label{eq:objective_gradient}
    \frac{\pp \mathcal{J}}{\pp \bm{\xi}_k} = 2 \sum_{i} (u(\vect{x}_i) - u_i^{\text{meas}}) \cdot q_k \cdot \grad_{\bm{\xi}} G(\vect{x}_i, \bm{\xi}_k)
\end{equation}

For the unit disk Green's function:
\begin{equation}
    \label{eq:greens_gradient}
    \grad_{\bm{\xi}} G_D(\vect{x}, \bm{\xi}) = \frac{1}{2\pi} \frac{\vect{x} - \bm{\xi}}{|\vect{x} - \bm{\xi}|^2} + \text{(image terms)}
\end{equation}

\subsubsection{Smoothness of Objective Function}

The choice between global and gradient-based methods depends on the smoothness of $\mathcal{J}$:
\begin{itemize}
    \item \textbf{BEM with analytical $G$}: $\mathcal{J}$ is smooth in $\bm{\xi}$ → gradient methods work well
    \item \textbf{FEM with nodal snapping}: $\mathcal{J}$ is piecewise constant in $\bm{\xi}$ → global methods required
    \item \textbf{FEM with barycentric interpolation}: $\mathcal{J}$ is smooth → gradient methods work
\end{itemize}

\subsection{Linear Algebraic Methods}

For the linear inverse problem \eqref{eq:linear_inverse}:

\begin{algorithm}[H]
\caption{Linear Inverse Solver with $L^1$ Regularization (IRLS)}
\label{alg:irls}
\begin{algorithmic}[1]
\Require Green's matrix $\mat{G}$, measurements $\vect{u}^{\text{meas}}$, regularization $\alpha$, tolerance $\epsilon$
\Ensure Source distribution $\vect{q}$
\State $\vect{q}^{(0)} \gets \vect{0}$
\For{$k = 0, 1, 2, \ldots$ until convergence}
    \State $W_{jj} \gets 1/(|q_j^{(k)}| + \epsilon)$
    \State $\vect{q}^{(k+1)} \gets (\mat{G}^\top\mat{G} + \alpha \mat{W})^{-1} \mat{G}^\top \vect{u}^{\text{meas}}$
    \If{$\norm{\vect{q}^{(k+1)} - \vect{q}^{(k)}} < \text{tol}$}
        \State \textbf{break}
    \EndIf
\EndFor
\State $\vect{q} \gets \vect{q}^{(k+1)} - \text{mean}(\vect{q}^{(k+1)})$
\end{algorithmic}
\end{algorithm}


% ============================================================================
% SECTION 9: IMPLEMENTATION
% ============================================================================
\section{Implementation Details}
\label{sec:implementation}

\subsection{Software Architecture}

The implementation consists of modular components:

\begin{enumerate}
    \item \texttt{forward\_solver.py}: FEM forward solver with DOLFINx
    \item \texttt{bem\_solver.py}: BEM solver with analytical Green's functions
    \item \texttt{conformal\_bem\_solver.py}: Conformal mapping + BEM for general domains
    \item \texttt{inverse\_solver.py}: Nonlinear and linear inverse solvers
    \item \texttt{parameter\_study.py}: Regularization parameter selection tools
\end{enumerate}

\subsection{FEM Implementation}

The FEM solver uses DOLFINx \citep{baratta2023dolfinx}, the latest version of the FEniCS project. Key implementation choices:
\begin{itemize}
    \item P1 (linear) Lagrange elements on triangular meshes
    \item Mesh generation via Gmsh \citep{geuzaine2009gmsh}
    \item Sparse direct solver (SciPy) for linear systems
    \item Zero-mean projection for null space handling
\end{itemize}

\subsection{BEM Implementation}

The BEM solver uses:
\begin{itemize}
    \item Analytical Green's function for unit disk
    \item NumPy vectorized operations for efficiency
    \item Uniform boundary discretization
\end{itemize}

\subsection{Configuration System}

A JSON configuration file controls all solver parameters:
\begin{lstlisting}[language=Python, caption=Example configuration]
{
    "problem": {"n_sources": 4, "noise_level": 0.001},
    "mesh": {"resolution": 0.05},
    "nonlinear": {
        "optimizer": "L-BFGS-B",
        "source_method": "interpolate"
    },
    "linear": {
        "regularization": "l1",
        "alpha": 1e-4
    }
}
\end{lstlisting}


% ============================================================================
% SECTION 10: CONCLUSIONS
% ============================================================================
\section{Conclusions and Future Work}
\label{sec:conclusions}

\subsection{Summary}

This report presented comprehensive mathematical foundations and numerical methods for inverse source localization in the Poisson equation. Key contributions include:

\begin{enumerate}
    \item Rigorous weak formulation handling Dirac delta sources
    \item FEM discretization with two source handling methods (snapping vs.\ interpolation)
    \item BEM formulation with analytical Green's functions
    \item Conformal mapping extension to general simply connected domains
    \item Comparison of regularization strategies ($L^2$, $L^1$, TV)
\end{enumerate}

\subsection{Method Comparison}

\begin{table}[H]
\centering
\caption{Comparison of forward solver approaches}
\label{tab:method_comparison}
\begin{tabular}{lccc}
\toprule
\textbf{Property} & \textbf{FEM} & \textbf{BEM (disk)} & \textbf{Conformal BEM} \\
\midrule
Domain flexibility & Any & Unit disk & Simply connected \\
Source positions & Grid/interpolated & Continuous & Continuous \\
Interior mesh & Required & Not needed & Not needed \\
Objective smoothness & Depends on method & Smooth & Smooth \\
Gradient computation & Numerical & Analytical & Analytical \\
\bottomrule
\end{tabular}
\end{table}

\begin{table}[H]
\centering
\caption{Comparison of regularization methods for point source recovery}
\label{tab:reg_comparison}
\begin{tabular}{lccc}
\toprule
\textbf{Property} & \textbf{$L^2$ (Tikhonov)} & \textbf{$L^1$} & \textbf{TV} \\
\midrule
Promotes & Smoothness & Sparsity & Piecewise constant \\
Point source recovery & Poor (diffuse) & Good & Poor \\
Optimization & Closed form & Convex (IRLS) & Convex (ADMM) \\
\bottomrule
\end{tabular}
\end{table}

\subsection{Future Directions}

\begin{itemize}
    \item Extension to 3D domains (BEM remains applicable; conformal mapping does not)
    \item Adaptive mesh refinement near recovered sources
    \item Uncertainty quantification for recovered parameters
    \item Real-time inverse solving for monitoring applications
\end{itemize}


% ============================================================================
% BIBLIOGRAPHY
% ============================================================================
\newpage
\bibliographystyle{abbrvnat}
\bibliography{references}

\end{document}
