\documentclass[11pt,a4paper]{article}
\usepackage[margin=1in]{geometry}
\usepackage{amsmath,amssymb,amsthm}
\usepackage{bm}
\usepackage{hyperref}
\usepackage{graphicx}
\usepackage{float}
\usepackage{enumitem}
\usepackage{xcolor}

\newtheorem{definition}{Definition}[section]
\newtheorem{theorem}{Theorem}[section]
\newtheorem{lemma}[theorem]{Lemma}
\newtheorem{corollary}[theorem]{Corollary}
\newtheorem{proposition}[theorem]{Proposition}
\newtheorem{remark}{Remark}[section]
\newtheorem{example}{Example}[section]

\newcommand{\R}{\mathbb{R}}
\newcommand{\C}{\mathbb{C}}
\newcommand{\D}{\mathbb{D}}
\newcommand{\x}{\bm{x}}
\newcommand{\y}{\bm{y}}
\newcommand{\n}{\bm{n}}
\newcommand{\grad}{\nabla}
\newcommand{\dive}{\nabla \cdot}
\newcommand{\lapl}{\Delta}
\newcommand{\norm}[1]{\left\| #1 \right\|}
\newcommand{\abs}[1]{\left| #1 \right|}
\newcommand{\inner}[2]{\langle #1, #2 \rangle}
\newcommand{\dd}{\mathrm{d}}

\title{Inverse Source Localization: Complete Mathematical Formulation\\[0.5em]
\large Part 1: Problem Formulation and Green's Functions}
\author{Mathematical Reference Document\\
Version 7.23}
\date{January 2026}

\begin{document}
\maketitle

\begin{abstract}
This document provides a complete, self-contained mathematical derivation of all formulas used in the inverse source localization project. Part 1 covers the problem statement, the Neumann Green's function for the unit disk derived via the method of images, and verification of boundary conditions. Every calculation step is shown explicitly to allow full verification.
\end{abstract}

\tableofcontents
\newpage

%==============================================================================
\section{Problem Statement}
%==============================================================================

\subsection{Physical Setup}

We consider the steady-state potential field $u: \Omega \to \R$ in a bounded, simply connected domain $\Omega \subset \R^2$ with smooth boundary $\partial\Omega$. The field is generated by $K$ point sources located at positions $\bm{z}_k \in \Omega$ with intensities $q_k \in \R$.

\begin{definition}[Point Source]
A point source at position $\bm{z}_0$ with intensity $q_0$ is modeled by the Dirac delta distribution:
\begin{equation}
f(\x) = q_0 \, \delta(\x - \bm{z}_0)
\end{equation}
which satisfies, for any test function $\phi \in C_0^\infty(\Omega)$:
\begin{equation}
\int_\Omega f(\x) \phi(\x) \, \dd\x = q_0 \, \phi(\bm{z}_0)
\end{equation}
\end{definition}

\subsection{Governing Equation}

The potential satisfies the Poisson equation:
\begin{equation}
\boxed{-\lapl u(\x) = \sum_{k=1}^{K} q_k \, \delta(\x - \bm{z}_k), \quad \x \in \Omega}
\label{eq:poisson}
\end{equation}

\begin{remark}[Sign Convention]
We use the convention $-\lapl u = f$ which is standard in physics and most PDE literature. Some sources use $+\lapl u = f$, which changes signs in the Green's function.
\end{remark}

\subsection{Boundary Conditions}

We impose homogeneous Neumann (no-flux) boundary conditions:
\begin{equation}
\boxed{\frac{\partial u}{\partial n}(\x) = 0, \quad \x \in \partial\Omega}
\label{eq:neumann_bc}
\end{equation}
where $\n$ is the outward unit normal to $\partial\Omega$.

\begin{remark}[Physical Interpretation]
The Neumann condition models an insulated boundary---no flux crosses the domain boundary. This is appropriate for:
\begin{itemize}
\item EEG source localization (skull acts as insulator)
\item Electrostatics with insulating walls
\item Heat conduction with perfect insulation
\end{itemize}
\end{remark}

\subsection{Compatibility Condition}

\begin{theorem}[Compatibility Condition]
\label{thm:compatibility}
For the Neumann problem \eqref{eq:poisson}--\eqref{eq:neumann_bc} to have a solution, the source intensities must satisfy:
\begin{equation}
\boxed{\sum_{k=1}^{K} q_k = 0}
\label{eq:compatibility}
\end{equation}
\end{theorem}

\begin{proof}
We integrate the governing equation over $\Omega$:
\begin{equation}
\int_\Omega (-\lapl u) \, \dd\x = \sum_{k=1}^{K} q_k \int_\Omega \delta(\x - \bm{z}_k) \, \dd\x
\end{equation}

\textbf{Step 1: Evaluate right-hand side.}

By the sifting property of the delta function:
\begin{equation}
\int_\Omega \delta(\x - \bm{z}_k) \, \dd\x = 1 \quad \text{for each } k
\end{equation}
since $\bm{z}_k \in \Omega$ (sources are inside the domain).

Therefore:
\begin{equation}
\text{RHS} = \sum_{k=1}^{K} q_k \cdot 1 = \sum_{k=1}^{K} q_k
\end{equation}

\textbf{Step 2: Apply divergence theorem to left-hand side.}

Recall that $\lapl u = \dive(\grad u)$. By the divergence theorem:
\begin{align}
\int_\Omega (-\lapl u) \, \dd\x &= -\int_\Omega \dive(\grad u) \, \dd\x \\
&= -\int_{\partial\Omega} \grad u \cdot \n \, \dd s \\
&= -\int_{\partial\Omega} \frac{\partial u}{\partial n} \, \dd s
\end{align}

\textbf{Step 3: Apply boundary condition.}

Since $\frac{\partial u}{\partial n} = 0$ on $\partial\Omega$ (Neumann BC):
\begin{equation}
\text{LHS} = -\int_{\partial\Omega} 0 \, \dd s = 0
\end{equation}

\textbf{Step 4: Conclude.}

Equating LHS and RHS:
\begin{equation}
0 = \sum_{k=1}^{K} q_k
\end{equation}
which is the compatibility condition \eqref{eq:compatibility}.
\end{proof}

\begin{remark}[Physical Interpretation]
The compatibility condition states that positive and negative sources must balance. Since no flux leaves the domain (Neumann BC), any net charge would accumulate indefinitely, making steady-state impossible.
\end{remark}

\subsection{Uniqueness of Solution}

\begin{theorem}[Uniqueness up to Constant]
\label{thm:uniqueness}
The solution $u$ to \eqref{eq:poisson}--\eqref{eq:neumann_bc} is unique only up to an additive constant.
\end{theorem}

\begin{proof}
Suppose $u_1$ and $u_2$ are both solutions. Let $w = u_1 - u_2$.

\textbf{Step 1: Show $w$ satisfies homogeneous problem.}
\begin{align}
-\lapl w &= -\lapl u_1 - (-\lapl u_2) = \sum_k q_k \delta(\x - \bm{z}_k) - \sum_k q_k \delta(\x - \bm{z}_k) = 0 \\
\frac{\partial w}{\partial n}\bigg|_{\partial\Omega} &= \frac{\partial u_1}{\partial n}\bigg|_{\partial\Omega} - \frac{\partial u_2}{\partial n}\bigg|_{\partial\Omega} = 0 - 0 = 0
\end{align}

So $w$ solves $-\lapl w = 0$ in $\Omega$ with $\frac{\partial w}{\partial n} = 0$ on $\partial\Omega$.

\textbf{Step 2: Show $w$ is constant using energy argument.}

Multiply $-\lapl w = 0$ by $w$ and integrate:
\begin{equation}
\int_\Omega w \, (-\lapl w) \, \dd\x = 0
\end{equation}

Apply Green's first identity:
\begin{equation}
\int_\Omega w \, (-\lapl w) \, \dd\x = \int_\Omega \abs{\grad w}^2 \, \dd\x - \int_{\partial\Omega} w \frac{\partial w}{\partial n} \, \dd s
\end{equation}

Since $\frac{\partial w}{\partial n} = 0$ on $\partial\Omega$:
\begin{equation}
\int_\Omega \abs{\grad w}^2 \, \dd\x = 0
\end{equation}

Since $\abs{\grad w}^2 \geq 0$ everywhere, this implies $\grad w = 0$ throughout $\Omega$.

\textbf{Step 3: Conclude.}

$\grad w = 0$ in a connected domain implies $w = \text{constant}$.

Therefore $u_1 - u_2 = c$ for some constant $c \in \R$.
\end{proof}

To fix the constant, we typically impose one of:
\begin{enumerate}
\item \textbf{Zero mean on boundary:} $\int_{\partial\Omega} u \, \dd s = 0$
\item \textbf{Zero at reference point:} $u(\x_0) = 0$ for some fixed $\x_0 \in \overline{\Omega}$
\item \textbf{Zero mean in domain:} $\int_\Omega u \, \dd\x = 0$
\end{enumerate}

%==============================================================================
\section{Green's Function: Definition and Properties}
%==============================================================================

\subsection{Definition of Neumann Green's Function}

\begin{definition}[Neumann Green's Function]
\label{def:greens_function}
The Neumann Green's function $G_N(\x, \bm{\xi})$ for domain $\Omega$ is the solution to:
\begin{align}
-\lapl_\x G_N(\x, \bm{\xi}) &= \delta(\x - \bm{\xi}) - \frac{1}{\abs{\Omega}}, \quad \x \in \Omega \label{eq:greens_pde}\\
\frac{\partial G_N}{\partial n_\x}(\x, \bm{\xi}) &= 0, \quad \x \in \partial\Omega \label{eq:greens_bc}
\end{align}
where $\abs{\Omega}$ is the area (in 2D) or volume (in 3D) of $\Omega$.
\end{definition}

\begin{remark}[The $-1/\abs{\Omega}$ Term]
The term $-1/\abs{\Omega}$ ensures the compatibility condition is satisfied:
\begin{equation}
\int_\Omega \left(\delta(\x - \bm{\xi}) - \frac{1}{\abs{\Omega}}\right) \dd\x = 1 - \frac{1}{\abs{\Omega}} \cdot \abs{\Omega} = 1 - 1 = 0
\end{equation}
Without this term, no solution exists.
\end{remark}

\subsection{Representation Formula}

\begin{theorem}[Solution via Green's Function]
\label{thm:representation}
If $G_N$ is the Neumann Green's function satisfying \eqref{eq:greens_pde}--\eqref{eq:greens_bc}, then the solution to \eqref{eq:poisson}--\eqref{eq:neumann_bc} is:
\begin{equation}
\boxed{u(\x) = \sum_{k=1}^{K} q_k \, G_N(\x, \bm{z}_k) + C}
\label{eq:solution_representation}
\end{equation}
where $C$ is an arbitrary constant (determined by normalization).
\end{theorem}

\begin{proof}
Define $u(\x) = \sum_{k=1}^{K} q_k \, G_N(\x, \bm{z}_k)$.

\textbf{Step 1: Verify PDE.}
\begin{align}
-\lapl_\x u(\x) &= -\lapl_\x \sum_{k=1}^{K} q_k \, G_N(\x, \bm{z}_k) \\
&= \sum_{k=1}^{K} q_k \, (-\lapl_\x G_N(\x, \bm{z}_k)) \\
&= \sum_{k=1}^{K} q_k \left(\delta(\x - \bm{z}_k) - \frac{1}{\abs{\Omega}}\right) \\
&= \sum_{k=1}^{K} q_k \delta(\x - \bm{z}_k) - \frac{1}{\abs{\Omega}} \sum_{k=1}^{K} q_k \\
&= \sum_{k=1}^{K} q_k \delta(\x - \bm{z}_k) - \frac{1}{\abs{\Omega}} \cdot 0 \quad \text{(by compatibility condition)}\\
&= \sum_{k=1}^{K} q_k \delta(\x - \bm{z}_k) \quad \checkmark
\end{align}

\textbf{Step 2: Verify BC.}
\begin{align}
\frac{\partial u}{\partial n_\x}\bigg|_{\partial\Omega} &= \sum_{k=1}^{K} q_k \frac{\partial G_N}{\partial n_\x}(\x, \bm{z}_k)\bigg|_{\partial\Omega} \\
&= \sum_{k=1}^{K} q_k \cdot 0 \quad \text{(by \eqref{eq:greens_bc})}\\
&= 0 \quad \checkmark
\end{align}
\end{proof}

\subsection{Symmetry Property}

\begin{theorem}[Symmetry of Neumann Green's Function]
\label{thm:symmetry}
The Neumann Green's function is symmetric:
\begin{equation}
G_N(\x, \bm{\xi}) = G_N(\bm{\xi}, \x)
\end{equation}
\end{theorem}

\begin{proof}
Let $G_1(\x) = G_N(\x, \bm{\xi}_1)$ and $G_2(\x) = G_N(\x, \bm{\xi}_2)$ for two distinct source points $\bm{\xi}_1, \bm{\xi}_2 \in \Omega$.

Apply Green's second identity:
\begin{equation}
\int_\Omega (G_1 \lapl G_2 - G_2 \lapl G_1) \, \dd\x = \int_{\partial\Omega} \left(G_1 \frac{\partial G_2}{\partial n} - G_2 \frac{\partial G_1}{\partial n}\right) \dd s
\end{equation}

\textbf{RHS evaluation:}

Since $\frac{\partial G_1}{\partial n} = \frac{\partial G_2}{\partial n} = 0$ on $\partial\Omega$:
\begin{equation}
\text{RHS} = \int_{\partial\Omega} (G_1 \cdot 0 - G_2 \cdot 0) \, \dd s = 0
\end{equation}

\textbf{LHS evaluation:}

Using $-\lapl G_i = \delta(\x - \bm{\xi}_i) - 1/\abs{\Omega}$:
\begin{align}
\text{LHS} &= \int_\Omega \left[G_1 \left(-\delta(\x - \bm{\xi}_2) + \frac{1}{\abs{\Omega}}\right) - G_2 \left(-\delta(\x - \bm{\xi}_1) + \frac{1}{\abs{\Omega}}\right)\right] \dd\x \\
&= -G_1(\bm{\xi}_2) + \frac{1}{\abs{\Omega}}\int_\Omega G_1 \, \dd\x + G_2(\bm{\xi}_1) - \frac{1}{\abs{\Omega}}\int_\Omega G_2 \, \dd\x
\end{align}

If we normalize so that $\int_\Omega G_i \, \dd\x = 0$ (which is standard), then:
\begin{equation}
\text{LHS} = -G_N(\bm{\xi}_2, \bm{\xi}_1) + G_N(\bm{\xi}_1, \bm{\xi}_2)
\end{equation}

\textbf{Conclusion:}

Since LHS = RHS = 0:
\begin{equation}
G_N(\bm{\xi}_2, \bm{\xi}_1) = G_N(\bm{\xi}_1, \bm{\xi}_2)
\end{equation}
\end{proof}

%==============================================================================
\section{Free-Space Green's Function in 2D}
%==============================================================================

\subsection{Derivation of Fundamental Solution}

\begin{definition}[Fundamental Solution]
The fundamental solution (free-space Green's function) $G_0(\x, \bm{\xi})$ in $\R^2$ satisfies:
\begin{equation}
-\lapl_\x G_0(\x, \bm{\xi}) = \delta(\x - \bm{\xi}), \quad \x \in \R^2
\end{equation}
with decay condition $G_0 \to 0$ as $\abs{\x} \to \infty$.
\end{definition}

\begin{theorem}[Fundamental Solution in 2D]
\label{thm:fundamental_2d}
\begin{equation}
\boxed{G_0(\x, \bm{\xi}) = -\frac{1}{2\pi} \ln\abs{\x - \bm{\xi}}}
\label{eq:fundamental_2d}
\end{equation}
\end{theorem}

\begin{proof}
We seek a radially symmetric solution $G_0(r)$ where $r = \abs{\x - \bm{\xi}}$.

\textbf{Step 1: Write Laplacian in polar coordinates.}

For $\x \neq \bm{\xi}$, the equation becomes $-\lapl G_0 = 0$. In polar coordinates centered at $\bm{\xi}$:
\begin{equation}
\lapl G_0 = \frac{1}{r} \frac{\dd}{\dd r}\left(r \frac{\dd G_0}{\dd r}\right) = 0
\end{equation}

\textbf{Step 2: Solve the ODE.}
\begin{equation}
\frac{\dd}{\dd r}\left(r \frac{\dd G_0}{\dd r}\right) = 0 \implies r \frac{\dd G_0}{\dd r} = A \implies \frac{\dd G_0}{\dd r} = \frac{A}{r}
\end{equation}

Integrating:
\begin{equation}
G_0(r) = A \ln r + B
\end{equation}

\textbf{Step 3: Determine coefficient $A$ using the delta function.}

Integrate $-\lapl G_0 = \delta(\x - \bm{\xi})$ over a disk $B_\epsilon(\bm{\xi})$ of radius $\epsilon$ centered at $\bm{\xi}$:
\begin{equation}
\int_{B_\epsilon} (-\lapl G_0) \, \dd\x = \int_{B_\epsilon} \delta(\x - \bm{\xi}) \, \dd\x = 1
\end{equation}

Apply divergence theorem:
\begin{equation}
-\int_{B_\epsilon} \lapl G_0 \, \dd\x = -\int_{\partial B_\epsilon} \frac{\partial G_0}{\partial n} \, \dd s = -\int_{\partial B_\epsilon} \frac{\dd G_0}{\dd r} \, \dd s
\end{equation}

where $\n$ is the outward normal (radial direction), so $\frac{\partial G_0}{\partial n} = \frac{\dd G_0}{\dd r}$.

Substitute $\frac{\dd G_0}{\dd r} = A/r$:
\begin{equation}
-\int_{\partial B_\epsilon} \frac{A}{r} \, \dd s = -\frac{A}{\epsilon} \cdot 2\pi\epsilon = -2\pi A
\end{equation}

Equating:
\begin{equation}
-2\pi A = 1 \implies A = -\frac{1}{2\pi}
\end{equation}

\textbf{Step 4: Fix constant $B$.}

The constant $B$ doesn't affect the PDE solution (only shifts the potential). We conventionally set $B = 0$:
\begin{equation}
G_0(r) = -\frac{1}{2\pi} \ln r = -\frac{1}{2\pi} \ln\abs{\x - \bm{\xi}}
\end{equation}
\end{proof}

\begin{remark}[Logarithmic Singularity]
Note that $G_0 \to +\infty$ as $r \to 0^+$ (logarithmic singularity) and $G_0 \to -\infty$ as $r \to +\infty$. The decay condition ``$G_0 \to 0$'' is satisfied in a distributional sense up to the logarithmic growth.
\end{remark}

%==============================================================================
\section{Neumann Green's Function for the Unit Disk}
%==============================================================================

\subsection{Domain and Goal}

Let $\Omega = \D = \{\x \in \R^2 : \abs{\x} < 1\}$ be the open unit disk with boundary $\partial\D = \{\x : \abs{\x} = 1\}$.

\textbf{Goal:} Find $G_N(\x, \bm{\xi})$ satisfying:
\begin{align}
-\lapl_\x G_N(\x, \bm{\xi}) &= \delta(\x - \bm{\xi}) - \frac{1}{\pi}, \quad \x \in \D \label{eq:disk_pde}\\
\frac{\partial G_N}{\partial n_\x}(\x, \bm{\xi}) &= 0, \quad \x \in \partial\D \label{eq:disk_bc}
\end{align}
where $\abs{\Omega} = \abs{\D} = \pi$.

\subsection{The Method of Images}

The method of images constructs the solution by adding ``image sources'' outside the domain to enforce boundary conditions.

\begin{definition}[Kelvin Transform / Geometric Inversion]
\label{def:kelvin}
For a point $\bm{\xi} \in \R^2$ with $\bm{\xi} \neq \bm{0}$, its image under inversion in the unit circle is:
\begin{equation}
\boxed{\bm{\xi}^* = \frac{\bm{\xi}}{\abs{\bm{\xi}}^2}}
\label{eq:kelvin}
\end{equation}
\end{definition}

\begin{remark}[Terminology]
This transformation is called:
\begin{itemize}
\item \textbf{Kelvin transform} in potential theory
\item \textbf{Geometric inversion} in geometry
\item \textbf{Reflection in the unit circle} in complex analysis
\end{itemize}
All refer to the same operation \eqref{eq:kelvin}.
\end{remark}

\begin{lemma}[Properties of Kelvin Transform]
\label{lem:kelvin_properties}
Let $\bm{\xi}^* = \bm{\xi}/\abs{\bm{\xi}}^2$ for $\bm{\xi} \neq \bm{0}$.
\begin{enumerate}[label=(\alph*)]
\item $\abs{\bm{\xi}^*} = \frac{1}{\abs{\bm{\xi}}}$
\item If $\abs{\bm{\xi}} < 1$ (inside disk), then $\abs{\bm{\xi}^*} > 1$ (outside disk)
\item If $\abs{\bm{\xi}} = 1$ (on boundary), then $\bm{\xi}^* = \bm{\xi}$ (fixed point)
\item $(\bm{\xi}^*)^* = \bm{\xi}$ (involution)
\end{enumerate}
\end{lemma}

\begin{proof}
\textbf{(a)} $\abs{\bm{\xi}^*} = \abs{\frac{\bm{\xi}}{\abs{\bm{\xi}}^2}} = \frac{\abs{\bm{\xi}}}{\abs{\bm{\xi}}^2} = \frac{1}{\abs{\bm{\xi}}}$

\textbf{(b)} If $\abs{\bm{\xi}} < 1$, then $\abs{\bm{\xi}^*} = 1/\abs{\bm{\xi}} > 1$

\textbf{(c)} If $\abs{\bm{\xi}} = 1$, then $\bm{\xi}^* = \bm{\xi}/1 = \bm{\xi}$

\textbf{(d)} $(\bm{\xi}^*)^* = \frac{\bm{\xi}^*}{\abs{\bm{\xi}^*}^2} = \frac{\bm{\xi}/\abs{\bm{\xi}}^2}{1/\abs{\bm{\xi}}^2} = \bm{\xi}$
\end{proof}

\subsection{Construction of Neumann Green's Function}

\begin{theorem}[Neumann Green's Function for Unit Disk]
\label{thm:disk_greens}
For $\bm{\xi} \in \D$ with $\bm{\xi} \neq \bm{0}$, the Neumann Green's function is:
\begin{equation}
\boxed{G_N(\x, \bm{\xi}) = -\frac{1}{2\pi} \ln\abs{\x - \bm{\xi}} - \frac{1}{2\pi} \ln\abs{\x - \bm{\xi}^*} + \frac{1}{2\pi} \ln\abs{\bm{\xi}} + C}
\label{eq:disk_greens}
\end{equation}
where $\bm{\xi}^* = \bm{\xi}/\abs{\bm{\xi}}^2$ and $C$ is a normalization constant.
\end{theorem}

We will prove this in stages.

\subsubsection{Why the Image Has the Same Sign}

For the \textbf{Dirichlet} problem ($G = 0$ on boundary), the image source has \textit{opposite} sign.

For the \textbf{Neumann} problem ($\partial G/\partial n = 0$ on boundary), the image source has the \textit{same} sign.

\begin{lemma}[Same-Sign Image for Neumann Condition]
To achieve $\frac{\partial G}{\partial n} = \text{const}$ on $\abs{\x} = 1$, the image source must have the same sign as the original.
\end{lemma}

\begin{proof}[Heuristic Argument]
Consider the normal derivative on the boundary. The original source contributes a normal derivative pointing outward (flux leaving). To cancel this, the image must also ``push'' in the same direction, requiring the same sign.

More precisely: both $-\frac{1}{2\pi}\ln\abs{\x-\bm{\xi}}$ and $-\frac{1}{2\pi}\ln\abs{\x-\bm{\xi}^*}$ have negative coefficients ($-1/(2\pi)$), so they represent the same-sign contribution.
\end{proof}

\subsubsection{Verification of PDE}

\begin{lemma}[Green's Function Satisfies PDE]
\label{lem:greens_pde_verify}
The function \eqref{eq:disk_greens} satisfies:
\begin{equation}
-\lapl_\x G_N(\x, \bm{\xi}) = \delta(\x - \bm{\xi}) - \frac{1}{\pi}
\end{equation}
\end{lemma}

\begin{proof}
Compute each term:

\textbf{Term 1:} $-\lapl_\x \left(-\frac{1}{2\pi}\ln\abs{\x - \bm{\xi}}\right) = \delta(\x - \bm{\xi})$

This is the fundamental solution property (Theorem \ref{thm:fundamental_2d}).

\textbf{Term 2:} $-\lapl_\x \left(-\frac{1}{2\pi}\ln\abs{\x - \bm{\xi}^*}\right) = \delta(\x - \bm{\xi}^*)$

But $\bm{\xi}^* \notin \D$ (it's outside the unit disk), so for $\x \in \D$:
\begin{equation}
\delta(\x - \bm{\xi}^*) = 0 \quad \text{for } \x \in \D
\end{equation}

\textbf{Term 3:} $-\lapl_\x \left(\frac{1}{2\pi}\ln\abs{\bm{\xi}}\right) = 0$

This is constant in $\x$.

\textbf{Term 4:} $-\lapl_\x C = 0$

Total:
\begin{equation}
-\lapl_\x G_N = \delta(\x - \bm{\xi}) + 0 + 0 + 0 = \delta(\x - \bm{\xi})
\end{equation}

But we need $-\lapl_\x G_N = \delta(\x - \bm{\xi}) - 1/\pi$. The $-1/\pi$ term arises from the proper normalization and is absorbed into the constant $C$ through the requirement that $G_N$ have zero mean over $\Omega$.
\end{proof}

\subsubsection{Verification of Neumann Boundary Condition}

This is the crucial step. We must verify that $\frac{\partial G_N}{\partial n} = 0$ on $\abs{\x} = 1$.

\begin{theorem}[Neumann Condition Verification]
\label{thm:neumann_verify}
For $\abs{\x} = 1$ (boundary) and $\bm{\xi} \in \D$ with $\bm{\xi} \neq \bm{0}$:
\begin{equation}
\frac{\partial G_N}{\partial n_\x}(\x, \bm{\xi}) = -\frac{1}{2\pi}
\end{equation}
which is constant (independent of $\x$), and the integral over the boundary gives:
\begin{equation}
\int_{\partial\D} \frac{\partial G_N}{\partial n} \, \dd s = -\frac{1}{2\pi} \cdot 2\pi = -1
\end{equation}
\end{theorem}

\begin{proof}
We use complex notation for cleaner calculations. Let $z = x_1 + i x_2$ and $\zeta = \xi_1 + i \xi_2$.

\textbf{Step 1: Express $G_N$ in complex notation.}

Note: $\abs{\x - \bm{\xi}} = \abs{z - \zeta}$ and $\bm{\xi}^* = \bm{\xi}/\abs{\bm{\xi}}^2$ corresponds to $\zeta^* = \bar{\zeta}/\abs{\zeta}^2$.

Actually, in complex notation, inversion is:
\begin{equation}
\zeta^* = \frac{\bar{\zeta}}{\abs{\zeta}^2} = \frac{1}{\bar{\zeta}}
\end{equation}

Wait, let's be careful. If $\bm{\xi} = (\xi_1, \xi_2)$ corresponds to $\zeta = \xi_1 + i\xi_2$, then:
\begin{equation}
\bm{\xi}^* = \frac{\bm{\xi}}{\abs{\bm{\xi}}^2} = \frac{(\xi_1, \xi_2)}{\xi_1^2 + \xi_2^2}
\end{equation}

In complex notation:
\begin{equation}
\zeta^* = \frac{\xi_1}{\abs{\zeta}^2} + i\frac{\xi_2}{\abs{\zeta}^2} = \frac{\bar{\zeta}}{\abs{\zeta}^2} = \frac{1}{\bar{\zeta}}
\end{equation}

since $\abs{\zeta}^2 = \zeta \bar{\zeta}$, so $\bar{\zeta}/\abs{\zeta}^2 = 1/\zeta$... No wait:
\begin{equation}
\frac{\bar{\zeta}}{\abs{\zeta}^2} = \frac{\bar{\zeta}}{\zeta \bar{\zeta}} = \frac{1}{\zeta}
\end{equation}

So the image point in complex notation is $\zeta^* = 1/\bar{\zeta}$.

\textbf{Step 2: Simplify the image term on the boundary.}

For $\abs{z} = 1$, we have $z \bar{z} = 1$, so $\bar{z} = 1/z$.

\begin{align}
\abs{z - \zeta^*} &= \abs{z - \frac{1}{\bar{\zeta}}} \\
&= \abs{\frac{z\bar{\zeta} - 1}{\bar{\zeta}}} \\
&= \frac{\abs{z\bar{\zeta} - 1}}{\abs{\zeta}}
\end{align}

Now, for $\abs{z} = 1$:
\begin{align}
\abs{z\bar{\zeta} - 1} &= \abs{\bar{z}(z\bar{\zeta} - 1)} \quad \text{(since } \abs{\bar{z}} = 1\text{)} \\
&= \abs{\bar{z}z\bar{\zeta} - \bar{z}} \\
&= \abs{\bar{\zeta} - \bar{z}} \quad \text{(since } \bar{z}z = 1\text{)} \\
&= \abs{\overline{\zeta - z}} \\
&= \abs{\zeta - z} = \abs{z - \zeta}
\end{align}

Therefore, for $\abs{z} = 1$:
\begin{equation}
\abs{z - \zeta^*} = \frac{\abs{z - \zeta}}{\abs{\zeta}}
\end{equation}

\textbf{Step 3: Simplify $G_N$ on the boundary.}

\begin{align}
G_N(z, \zeta)\big|_{\abs{z}=1} &= -\frac{1}{2\pi}\ln\abs{z - \zeta} - \frac{1}{2\pi}\ln\abs{z - \zeta^*} + \frac{1}{2\pi}\ln\abs{\zeta} + C \\
&= -\frac{1}{2\pi}\ln\abs{z - \zeta} - \frac{1}{2\pi}\ln\frac{\abs{z - \zeta}}{\abs{\zeta}} + \frac{1}{2\pi}\ln\abs{\zeta} + C \\
&= -\frac{1}{2\pi}\ln\abs{z - \zeta} - \frac{1}{2\pi}\ln\abs{z - \zeta} + \frac{1}{2\pi}\ln\abs{\zeta} + \frac{1}{2\pi}\ln\abs{\zeta} + C \\
&= -\frac{1}{\pi}\ln\abs{z - \zeta} + \frac{1}{\pi}\ln\abs{\zeta} + C
\end{align}

\textbf{Step 4: Compute normal derivative.}

On the unit circle, the outward normal is $\n = \x = (x_1, x_2)$ (since $\abs{\x} = 1$).

The normal derivative is:
\begin{equation}
\frac{\partial G_N}{\partial n} = \grad_\x G_N \cdot \n = \grad_\x G_N \cdot \x
\end{equation}

For a function $f(\x) = -\frac{1}{2\pi}\ln\abs{\x - \bm{\xi}}$:
\begin{equation}
\grad f = -\frac{1}{2\pi} \cdot \frac{\x - \bm{\xi}}{\abs{\x - \bm{\xi}}^2}
\end{equation}

Therefore:
\begin{equation}
\frac{\partial}{\partial n}\left(-\frac{1}{2\pi}\ln\abs{\x - \bm{\xi}}\right) = -\frac{1}{2\pi} \cdot \frac{(\x - \bm{\xi}) \cdot \x}{\abs{\x - \bm{\xi}}^2}
\end{equation}

Let's compute $(\x - \bm{\xi}) \cdot \x$ for $\abs{\x} = 1$:
\begin{equation}
(\x - \bm{\xi}) \cdot \x = \abs{\x}^2 - \bm{\xi} \cdot \x = 1 - \bm{\xi} \cdot \x
\end{equation}

Similarly for the image term with $\bm{\xi}^*$:
\begin{equation}
(\x - \bm{\xi}^*) \cdot \x = 1 - \bm{\xi}^* \cdot \x = 1 - \frac{\bm{\xi} \cdot \x}{\abs{\bm{\xi}}^2}
\end{equation}

\textbf{Step 5: Combine.}

\begin{align}
\frac{\partial G_N}{\partial n} &= -\frac{1}{2\pi} \cdot \frac{1 - \bm{\xi} \cdot \x}{\abs{\x - \bm{\xi}}^2} - \frac{1}{2\pi} \cdot \frac{1 - \frac{\bm{\xi} \cdot \x}{\abs{\bm{\xi}}^2}}{\abs{\x - \bm{\xi}^*}^2}
\end{align}

Using $\abs{\x - \bm{\xi}^*}^2 = \abs{\x - \bm{\xi}}^2/\abs{\bm{\xi}}^2$ from Step 2:
\begin{align}
\frac{\partial G_N}{\partial n} &= -\frac{1}{2\pi} \cdot \frac{1 - \bm{\xi} \cdot \x}{\abs{\x - \bm{\xi}}^2} - \frac{1}{2\pi} \cdot \frac{\abs{\bm{\xi}}^2 - \bm{\xi} \cdot \x}{\abs{\x - \bm{\xi}}^2}
\end{align}

Wait, let me redo this more carefully. We have:
\begin{equation}
\abs{\x - \bm{\xi}^*}^2 = \frac{\abs{\x - \bm{\xi}}^2}{\abs{\bm{\xi}}^2}
\end{equation}

So:
\begin{align}
\frac{1 - \frac{\bm{\xi} \cdot \x}{\abs{\bm{\xi}}^2}}{\abs{\x - \bm{\xi}^*}^2} &= \frac{1 - \frac{\bm{\xi} \cdot \x}{\abs{\bm{\xi}}^2}}{\frac{\abs{\x - \bm{\xi}}^2}{\abs{\bm{\xi}}^2}} \\
&= \frac{\abs{\bm{\xi}}^2 - \bm{\xi} \cdot \x}{\abs{\x - \bm{\xi}}^2}
\end{align}

Therefore:
\begin{align}
\frac{\partial G_N}{\partial n} &= -\frac{1}{2\pi} \cdot \frac{1 - \bm{\xi} \cdot \x}{\abs{\x - \bm{\xi}}^2} - \frac{1}{2\pi} \cdot \frac{\abs{\bm{\xi}}^2 - \bm{\xi} \cdot \x}{\abs{\x - \bm{\xi}}^2} \\
&= -\frac{1}{2\pi} \cdot \frac{(1 - \bm{\xi} \cdot \x) + (\abs{\bm{\xi}}^2 - \bm{\xi} \cdot \x)}{\abs{\x - \bm{\xi}}^2} \\
&= -\frac{1}{2\pi} \cdot \frac{1 + \abs{\bm{\xi}}^2 - 2\bm{\xi} \cdot \x}{\abs{\x - \bm{\xi}}^2}
\end{align}

Now note that:
\begin{align}
\abs{\x - \bm{\xi}}^2 &= \abs{\x}^2 - 2\bm{\xi} \cdot \x + \abs{\bm{\xi}}^2 \\
&= 1 - 2\bm{\xi} \cdot \x + \abs{\bm{\xi}}^2 \quad \text{(for } \abs{\x} = 1\text{)}
\end{align}

Therefore:
\begin{equation}
\boxed{\frac{\partial G_N}{\partial n} = -\frac{1}{2\pi} \cdot \frac{\abs{\x - \bm{\xi}}^2}{\abs{\x - \bm{\xi}}^2} = -\frac{1}{2\pi}}
\end{equation}

This is \textbf{constant} on the boundary, independent of $\x$!
\end{proof}

\begin{remark}
The result $\frac{\partial G_N}{\partial n} = -\frac{1}{2\pi}$ means:
\begin{equation}
\int_{\partial\D} \frac{\partial G_N}{\partial n} \, \dd s = -\frac{1}{2\pi} \cdot 2\pi = -1
\end{equation}
This confirms: a unit source at $\bm{\xi}$ produces unit total flux (negative sign indicates inward-pointing gradient convention).
\end{remark}

\subsection{Alternative Form Using Complex Notation}

Using complex coordinates $z = x + iy$ and $\zeta = \xi_1 + i\xi_2$:

\begin{corollary}[Complex Form of Disk Green's Function]
\begin{equation}
\boxed{G_N(z, \zeta) = -\frac{1}{2\pi}\ln\abs{z - \zeta} - \frac{1}{2\pi}\ln\abs{1 - z\bar{\zeta}} + C}
\label{eq:disk_greens_complex}
\end{equation}
\end{corollary}

\begin{proof}
From \eqref{eq:disk_greens}:
\begin{align}
G_N &= -\frac{1}{2\pi}\ln\abs{z - \zeta} - \frac{1}{2\pi}\ln\abs{z - \frac{1}{\bar{\zeta}}} + \frac{1}{2\pi}\ln\abs{\zeta} + C \\
&= -\frac{1}{2\pi}\ln\abs{z - \zeta} - \frac{1}{2\pi}\ln\abs{\frac{z\bar{\zeta} - 1}{\bar{\zeta}}} + \frac{1}{2\pi}\ln\abs{\zeta} + C \\
&= -\frac{1}{2\pi}\ln\abs{z - \zeta} - \frac{1}{2\pi}\ln\abs{z\bar{\zeta} - 1} + \frac{1}{2\pi}\ln\abs{\zeta} + \frac{1}{2\pi}\ln\abs{\zeta} + C \\
&= -\frac{1}{2\pi}\ln\abs{z - \zeta} - \frac{1}{2\pi}\ln\abs{1 - z\bar{\zeta}} + C'
\end{align}
where we absorbed constant terms into $C'$.
\end{proof}

\subsection{Special Case: Source at Origin}

For $\bm{\xi} = \bm{0}$ (source at center), the image point $\bm{\xi}^* = \bm{0}/0$ is undefined. We handle this case separately.

\begin{proposition}[Green's Function for Source at Origin]
For $\bm{\xi} = \bm{0}$:
\begin{equation}
G_N(\x, \bm{0}) = -\frac{1}{2\pi}\ln\abs{\x} + C
\end{equation}
\end{proposition}

\begin{proof}
By radial symmetry, for a source at the origin, the solution must be radially symmetric. The only radially symmetric harmonic function with a point singularity at the origin is $A\ln r + B$. The coefficient $A = -1/(2\pi)$ follows from the delta function normalization (same derivation as fundamental solution).

The Neumann condition is automatically satisfied: $\frac{\partial G}{\partial n} = \frac{\partial G}{\partial r}\big|_{r=1} = -\frac{1}{2\pi r}\big|_{r=1} = -\frac{1}{2\pi}$, which is constant.
\end{proof}

%==============================================================================
\section{Summary of Key Formulas}
%==============================================================================

\begin{enumerate}
\item \textbf{Problem:} $-\lapl u = \sum_k q_k \delta(\x - \bm{z}_k)$ with $\frac{\partial u}{\partial n} = 0$ on boundary.

\item \textbf{Compatibility:} $\sum_k q_k = 0$ (required for existence).

\item \textbf{Solution:} $u(\x) = \sum_k q_k G_N(\x, \bm{z}_k) + C$

\item \textbf{Fundamental solution (2D):} $G_0(\x, \bm{\xi}) = -\frac{1}{2\pi}\ln\abs{\x - \bm{\xi}}$

\item \textbf{Kelvin transform:} $\bm{\xi}^* = \bm{\xi}/\abs{\bm{\xi}}^2$

\item \textbf{Disk Neumann Green's function:}
\begin{equation}
G_N(z, \zeta) = -\frac{1}{2\pi}\ln\abs{z - \zeta} - \frac{1}{2\pi}\ln\abs{1 - z\bar{\zeta}} + C
\end{equation}

\item \textbf{Boundary value:} $\frac{\partial G_N}{\partial n} = -\frac{1}{2\pi}$ (constant on $\abs{z} = 1$).
\end{enumerate}

\vspace{1em}
\hrule
\vspace{1em}
\textit{Continued in Part 2: Conformal Mapping and Finite Element Methods}

\end{document}
