\documentclass[11pt,a4paper]{article}

% ============================================================================
% PACKAGES
% ============================================================================
\usepackage[utf8]{inputenc}
\usepackage[T1]{fontenc}
\usepackage{amsmath,amssymb,amsthm}
\usepackage{mathtools}
\usepackage{bm}
\usepackage{geometry}
\usepackage{graphicx}
\usepackage{float}
\usepackage{booktabs}
\usepackage{array}
\usepackage{algorithm}
\usepackage{algpseudocode}
\usepackage{listings}
\usepackage{xcolor}
\usepackage{hyperref}
\usepackage{cleveref}
\usepackage{natbib}
\usepackage{caption}
\usepackage{subcaption}
\usepackage{tikz}
\usetikzlibrary{shapes,arrows,positioning}

% ============================================================================
% PAGE SETUP
% ============================================================================
\geometry{margin=1in}
\setlength{\parindent}{0pt}
\setlength{\parskip}{6pt}

% ============================================================================
% THEOREM ENVIRONMENTS
% ============================================================================
\theoremstyle{plain}
\newtheorem{theorem}{Theorem}[section]
\newtheorem{lemma}[theorem]{Lemma}
\newtheorem{proposition}[theorem]{Proposition}
\newtheorem{corollary}[theorem]{Corollary}

\theoremstyle{definition}
\newtheorem{definition}[theorem]{Definition}
\newtheorem{example}[theorem]{Example}
\newtheorem{remark}[theorem]{Remark}
\newtheorem{problem}{Problem}[section]

% ============================================================================
% CUSTOM COMMANDS
% ============================================================================
\newcommand{\R}{\mathbb{R}}
\newcommand{\C}{\mathbb{C}}
\newcommand{\N}{\mathbb{N}}
\newcommand{\D}{\mathbb{D}}
\newcommand{\boundary}{\partial\Omega}
\newcommand{\inner}[2]{\langle #1, #2 \rangle}
\newcommand{\norm}[1]{\left\| #1 \right\|}
\newcommand{\abs}[1]{\left| #1 \right|}
\newcommand{\dd}{\mathrm{d}}
\newcommand{\pp}{\partial}
\newcommand{\grad}{\nabla}
\newcommand{\divergence}{\nabla \cdot}
\newcommand{\laplacian}{\Delta}
\newcommand{\vect}[1]{\mathbf{#1}}
\newcommand{\mat}[1]{\mathbf{#1}}
\newcommand{\dirac}{\delta}
\newcommand{\argmin}{\operatorname{argmin}}

% Code listing style
\lstset{
    basicstyle=\small\ttfamily,
    keywordstyle=\color{blue},
    commentstyle=\color{gray},
    stringstyle=\color{red},
    numbers=left,
    numberstyle=\tiny,
    numbersep=5pt,
    frame=single,
    breaklines=true
}

% ============================================================================
% DOCUMENT
% ============================================================================
\begin{document}

% ============================================================================
% TITLE PAGE
% ============================================================================
\begin{titlepage}
    \centering
    \vspace*{2cm}
    
    {\Huge\bfseries Inverse Source Localization\\[0.5cm] for the Poisson Equation}
    
    \vspace{1cm}
    
    {\Large Mathematical Foundations and Numerical Methods}
    
    \vspace{2cm}
    
    {\large Technical Report}
    
    \vspace{1cm}
    
    {\large January 2026}
    
    \vfill
    
    \begin{abstract}
    \noindent
    This report presents a comprehensive mathematical framework for inverse source localization problems governed by the Poisson equation with Neumann boundary conditions. We develop both the forward problem---computing boundary measurements from interior point sources---and the inverse problem---recovering source locations and intensities from boundary data. The central theoretical contribution is a rigorous treatment of \emph{conformal invariance}: we prove that for any simply connected planar domain $\Omega$ conformally equivalent to the unit disk, the solution can be expressed as $u(z) = \sum_k q_k G_{\D}(f(z), f(z_k))$, where $G_{\D}$ is the disk Green's function and $f: \Omega \to \D$ is the conformal map. Crucially, source intensities are \emph{preserved} under conformal transformation. Three complementary numerical approaches are presented and compared: (1) an \emph{analytical solver} using the closed-form Neumann Green's function for the unit disk derived via the method of images, (2) a \emph{Boundary Element Method} (BEM) with numerical integration for validation, and (3) a \emph{Finite Element Method} (FEM) for arbitrary meshed domains. For the inverse problem, we formulate both nonlinear optimization (for continuous source positions) and linear algebraic approaches (for discretized source grids) with various regularization strategies including Tikhonov ($L^2$), sparsity-promoting ($L^1$), and Total Variation (TV). The L-curve method provides automatic regularization parameter selection. Applications to EEG source localization motivate the inclusion of brain-like domain geometries.
    \end{abstract}
    
\end{titlepage}

% ============================================================================
% TABLE OF CONTENTS
% ============================================================================
\tableofcontents
\newpage

% ============================================================================
% SECTION 1: INTRODUCTION
% ============================================================================
\section{Introduction}
\label{sec:introduction}

The inverse source problem for elliptic partial differential equations arises in numerous applications including electroencephalography (EEG) source localization \citep{grech2008review}, geophysical prospecting \citep{isakov2006inverse}, medical imaging \citep{ammari2004reconstruction}, environmental monitoring \citep{el2005identification}, and non-destructive testing \citep{andrieux2006inverse}. The fundamental challenge is to determine the location and strength of interior sources from measurements taken only on the boundary of the domain.

\subsection{Problem Overview}

Consider a bounded domain $\Omega \subset \R^2$ with boundary $\boundary$. The forward problem consists of solving the Poisson equation with point sources:
\begin{equation}
    \label{eq:forward_intro}
    -\laplacian u = \sum_{k=1}^{N} q_k \dirac(\vect{x} - \bm{\xi}_k) \quad \text{in } \Omega
\end{equation}
subject to appropriate boundary conditions, where $\bm{\xi}_k \in \Omega$ are source locations and $q_k \in \R$ are source intensities.

The inverse problem seeks to recover the source parameters $\{(\bm{\xi}_k, q_k)\}_{k=1}^{N}$ from boundary measurements of $u$ or its derivatives.

\subsection{Main Contributions}

The key theoretical contribution of this work is a rigorous treatment of \textbf{conformal invariance}. We prove that:
\begin{enumerate}
    \item The Poisson equation with point sources transforms covariantly under conformal maps
    \item Source intensities are \emph{preserved} (not scaled) under the transformation
    \item The Green's function for any simply connected domain can be expressed using the disk Green's function composed with the conformal map
\end{enumerate}

This yields a practical algorithm: for \emph{any} simply connected domain $\Omega$ with conformal map $f: \Omega \to \D$, the solution is:
\begin{equation}
    u(z) = \sum_{k=1}^{N} q_k \, G_{\D}(f(z), f(z_k))
\end{equation}

\subsection{Scope and Organization}

This report is organized as follows:
\begin{itemize}
    \item \textbf{Section~\ref{sec:forward}}: Mathematical formulation of the forward problem
    \item \textbf{Section~\ref{sec:analytical}}: Analytical solution for the unit disk
    \item \textbf{Section~\ref{sec:bem}}: Boundary Element Method with numerical integration
    \item \textbf{Section~\ref{sec:fem}}: Finite Element Method discretization
    \item \textbf{Section~\ref{sec:conformal}}: Conformal mapping for general domains (with rigorous proofs)
    \item \textbf{Section~\ref{sec:inverse}}: Inverse problem formulations
    \item \textbf{Section~\ref{sec:regularization}}: Regularization methods and parameter selection
    \item \textbf{Section~\ref{sec:numerical}}: Numerical experiments and comparison
    \item \textbf{Section~\ref{sec:implementation}}: Implementation details
\end{itemize}


% ============================================================================
% SECTION 2: FORWARD PROBLEM
% ============================================================================
\section{The Forward Problem}
\label{sec:forward}

\subsection{Strong Formulation}

Let $\Omega \subset \R^2$ be a bounded, simply connected domain with smooth boundary $\boundary$. We consider the Poisson equation with point sources and homogeneous Neumann boundary conditions:

\begin{problem}[Strong Form]
\label{prob:strong}
Find $u: \Omega \to \R$ such that
\begin{align}
    -\laplacian u &= f \quad \text{in } \Omega \label{eq:poisson}\\
    \frac{\pp u}{\pp \vect{n}} &= 0 \quad \text{on } \boundary \label{eq:neumann}
\end{align}
where $\vect{n}$ is the outward unit normal to $\boundary$, and the source term is
\begin{equation}
    \label{eq:source_term}
    f(\vect{x}) = \sum_{k=1}^{N} q_k \dirac(\vect{x} - \bm{\xi}_k)
\end{equation}
with source locations $\bm{\xi}_k \in \Omega$ and intensities $q_k \in \R$.
\end{problem}

\subsection{Compatibility Condition}

For the Neumann problem to admit a solution, the source term must satisfy a compatibility condition. Integrating \eqref{eq:poisson} over $\Omega$ and applying the divergence theorem:
\begin{equation}
    -\int_\Omega \laplacian u \, \dd\vect{x} = -\int_{\boundary} \frac{\pp u}{\pp \vect{n}} \, \dd s = 0
\end{equation}

Thus, we require:
\begin{equation}
    \label{eq:compatibility}
    \boxed{\sum_{k=1}^{N} q_k = 0}
\end{equation}

\begin{remark}[Physical Interpretation]
The compatibility condition \eqref{eq:compatibility} states that the total source strength must equal the total sink strength. This reflects conservation: with no flux through the boundary, what flows out of sources must flow into sinks.
\end{remark}

\subsection{Uniqueness}

The solution to Problem~\ref{prob:strong} is unique only up to an additive constant. We fix this ambiguity by imposing:
\begin{equation}
    \label{eq:zero_mean}
    \int_\Omega u \, \dd\vect{x} = 0
\end{equation}

\subsection{Weak Formulation}

The presence of Dirac delta distributions in \eqref{eq:source_term} requires a weak (variational) formulation. Let $H^1(\Omega)$ denote the Sobolev space of functions with square-integrable weak derivatives.

Multiplying \eqref{eq:poisson} by a test function $v \in H^1(\Omega)$ and integrating by parts:
\begin{align}
    \int_\Omega (-\laplacian u) v \, \dd\vect{x} &= \int_\Omega f v \, \dd\vect{x} \\
    \int_\Omega \grad u \cdot \grad v \, \dd\vect{x} - \int_{\boundary} \frac{\pp u}{\pp \vect{n}} v \, \dd s &= \int_\Omega f v \, \dd\vect{x}
\end{align}

With the Neumann condition \eqref{eq:neumann}, the boundary integral vanishes.

\begin{problem}[Weak Form]
\label{prob:weak}
Find $u \in H^1(\Omega)$ with $\int_\Omega u \, \dd\vect{x} = 0$ such that for all $v \in H^1(\Omega)$:
\begin{equation}
    \label{eq:weak_form}
    \int_\Omega \grad u \cdot \grad v \, \dd\vect{x} = \sum_{k=1}^{N} q_k \, v(\bm{\xi}_k)
\end{equation}
\end{problem}

\subsection{Green's Function Representation}

The solution to Problem~\ref{prob:strong} can be expressed using the Neumann Green's function.

\begin{definition}[Neumann Green's Function]
\label{def:greens}
The Neumann Green's function $G: \Omega \times \Omega \to \R$ satisfies:
\begin{align}
    -\laplacian_{\vect{x}} G(\vect{x}, \bm{\xi}) &= \dirac(\vect{x} - \bm{\xi}) - \frac{1}{|\Omega|} \quad \text{in } \Omega \label{eq:greens_pde}\\
    \frac{\pp G}{\pp \vect{n}_{\vect{x}}} &= 0 \quad \text{on } \boundary \label{eq:greens_bc}\\
    \int_\Omega G(\vect{x}, \bm{\xi}) \, \dd\vect{x} &= 0 \label{eq:greens_norm}
\end{align}
\end{definition}

The solution to Problem~\ref{prob:strong} is:
\begin{equation}
    \label{eq:greens_representation}
    u(\vect{x}) = \sum_{k=1}^{N} q_k \, G(\vect{x}, \bm{\xi}_k)
\end{equation}


% ============================================================================
% SECTION 3: ANALYTICAL SOLUTION
% ============================================================================
\section{Analytical Solution for the Unit Disk}
\label{sec:analytical}

For the unit disk $\D = \{z \in \C : |z| < 1\}$, the Neumann Green's function admits an explicit closed-form expression using the method of images \citep{jackson1999classical,stakgold2011greens}.

\subsection{The Kelvin Transform}

\begin{definition}[Kelvin Transform]
For a point $\xi \in \D$ with $\xi \neq 0$, the \textbf{Kelvin transform} (geometric inversion) with respect to the unit circle is:
\begin{equation}
    \xi^* = \frac{\xi}{|\xi|^2} = \frac{\bar{\xi}}{|\xi|^2}
\end{equation}
This maps interior points to exterior points: if $|\xi| < 1$, then $|\xi^*| = 1/|\xi| > 1$.
\end{definition}

The Kelvin transform has a crucial geometric property:

\begin{lemma}[Boundary Identity]
\label{lem:boundary_identity}
For any $x$ on the unit circle ($|x| = 1$) and any $\xi \in \D$ with $\xi \neq 0$:
\begin{equation}
    |x - \xi^*| = \frac{|x - \xi|}{|\xi|}
\end{equation}
\end{lemma}

\begin{proof}
Let $|x| = 1$. We compute $|x - \xi^*|^2$ directly:
\begin{align}
    |x - \xi^*|^2 &= \left|x - \frac{\bar{\xi}}{|\xi|^2}\right|^2 
    = \left(x - \frac{\bar{\xi}}{|\xi|^2}\right)\overline{\left(x - \frac{\bar{\xi}}{|\xi|^2}\right)} \\
    &= |x|^2 - \frac{x\xi}{|\xi|^2} - \frac{\bar{x}\bar{\xi}}{|\xi|^2} + \frac{|\xi|^2}{|\xi|^4} \\
    &= 1 - \frac{x\xi + \bar{x}\bar{\xi}}{|\xi|^2} + \frac{1}{|\xi|^2}
\end{align}

Similarly, for $|x - \xi|^2$:
\begin{equation}
    |x - \xi|^2 = |x|^2 - x\bar{\xi} - \bar{x}\xi + |\xi|^2 = 1 - (x\bar{\xi} + \bar{x}\xi) + |\xi|^2
\end{equation}

Therefore:
\begin{equation}
    \frac{|x - \xi|^2}{|\xi|^2} = \frac{1}{|\xi|^2} - \frac{x\bar{\xi} + \bar{x}\xi}{|\xi|^2} + 1 = |x - \xi^*|^2
\end{equation}
Taking square roots yields the result.
\end{proof}

\subsection{Construction of the Green's Function}

\begin{theorem}[Neumann Green's Function for the Unit Disk]
\label{thm:disk_greens}
The Neumann Green's function for the unit disk is:
\begin{equation}
    \label{eq:disk_greens}
    \boxed{G_{\D}(x, \xi) = -\frac{1}{2\pi}\left[\ln|x - \xi| + \ln|x - \xi^*| - \ln|\xi|\right]}
\end{equation}
where $\xi^* = \xi/|\xi|^2$ is the Kelvin reflection of $\xi$.
\end{theorem}

\begin{proof}
We verify that $G_{\D}$ satisfies all required properties.

\textbf{Step 1: Fundamental solution property.}
The term $-\frac{1}{2\pi}\ln|x - \xi|$ is the fundamental solution of the 2D Laplacian:
\begin{equation}
    -\laplacian_x \left(-\frac{1}{2\pi}\ln|x - \xi|\right) = \dirac(x - \xi)
\end{equation}

\textbf{Step 2: Image term is harmonic.}
Since $|\xi| < 1$ implies $|\xi^*| = 1/|\xi| > 1$, the image point $\xi^*$ lies \emph{outside} the disk. Therefore, $-\frac{1}{2\pi}\ln|x - \xi^*|$ is harmonic for all $x \in \D$.

\textbf{Step 3: Neumann boundary condition.}
On $\pp\D$ where $|x| = 1$, the outward normal is $\vect{n} = x$ (identifying $\R^2 \cong \C$). The normal derivative of the logarithm is:
\begin{equation}
    \frac{\pp}{\pp n}\ln|x - \xi| = \frac{(x - \xi) \cdot x}{|x - \xi|^2} = \frac{|x|^2 - \text{Re}(\bar{x}\xi)}{|x - \xi|^2} = \frac{1 - \text{Re}(\bar{x}\xi)}{|x - \xi|^2}
\end{equation}

For the image term, using \cref{lem:boundary_identity}:
\begin{equation}
    \frac{\pp}{\pp n}\ln|x - \xi^*| = \frac{1 - \text{Re}(\bar{x}\xi^*)}{|x - \xi^*|^2} = \frac{1 - \text{Re}(\bar{x}\xi^*)}{|x - \xi|^2/|\xi|^2}
\end{equation}

Since $\text{Re}(\bar{x}\xi^*) = \text{Re}(\bar{x}\bar{\xi}/|\xi|^2) = \text{Re}(\bar{x}\xi)/|\xi|^2$, combining the terms shows:
\begin{equation}
    \frac{\pp G_{\D}}{\pp n}\bigg|_{\pp\D} = -\frac{1}{2\pi}
\end{equation}
which is constant, as required for the Neumann problem.

\textbf{Step 4: Normalization check.}
The integral of the normal derivative over the boundary is:
\begin{equation}
    \int_{\pp\D} \frac{\pp G_{\D}}{\pp n} \, ds = -\frac{1}{2\pi} \cdot 2\pi = -1
\end{equation}
confirming the correct normalization for a unit source.
\end{proof}

\begin{remark}[Source at Origin]
When $\xi = 0$, the image point is undefined. Taking the limit $\xi \to 0$:
\begin{equation}
    G_{\D}(x, 0) = -\frac{1}{2\pi}\ln|x|
\end{equation}
\end{remark}

\subsection{Boundary Simplification}

\begin{corollary}
On the boundary $|x| = 1$, using \cref{lem:boundary_identity}:
\begin{equation}
    G_{\D}(x, \xi)\big|_{|x|=1} = -\frac{1}{2\pi}\left[\ln|x - \xi| + \ln\frac{|x-\xi|}{|\xi|} - \ln|\xi|\right] = -\frac{1}{\pi}\ln|x - \xi|
\end{equation}
\end{corollary}

Thus the boundary solution for sources at $\xi_k$ with intensities $q_k$ is:
\begin{equation}
    \label{eq:boundary_solution}
    u(x)\big|_{|x|=1} = -\frac{1}{\pi}\sum_{k=1}^{N} q_k \ln|x - \xi_k|
\end{equation}

\subsection{Analytical Forward Solver}

Using \eqref{eq:disk_greens}, the forward problem has the explicit solution:
\begin{equation}
    \label{eq:analytical_forward}
    u(\vect{x}) = -\frac{1}{2\pi} \sum_{k=1}^{N} q_k \left[\ln|\vect{x} - \bm{\xi}_k| + \ln|\vect{x} - \bm{\xi}_k^*| - \ln|\bm{\xi}_k|\right]
\end{equation}

\textbf{Key advantages}:
\begin{itemize}
    \item Source positions $\bm{\xi}_k$ appear continuously---no mesh discretization required
    \item Exact evaluation at any boundary point
    \item Analytical gradients available for optimization
\end{itemize}


% ============================================================================
% SECTION 4: BOUNDARY ELEMENT METHOD
% ============================================================================
\section{Boundary Element Method}
\label{sec:bem}

The Boundary Element Method (BEM) provides an alternative approach using numerical integration of the boundary integral equation \citep{sauter2011boundary,steinbach2008numerical}.

\subsection{Fundamental Solution}

The fundamental solution for the 2D Laplacian is:
\begin{equation}
    \label{eq:fundamental}
    \Phi(\vect{x}, \bm{\xi}) = -\frac{1}{2\pi} \ln|\vect{x} - \bm{\xi}|
\end{equation}

\subsection{BEM Discretization}

We discretize the boundary into $n_e$ elements with collocation points at element midpoints. For point sources, the BEM evaluates:
\begin{equation}
    \label{eq:bem_forward}
    u(\vect{x}_i) = \sum_{k=1}^{N} q_k \, G_{\text{BEM}}(\vect{x}_i, \bm{\xi}_k)
\end{equation}
where $G_{\text{BEM}}$ is computed via numerical integration using Gaussian quadrature.

\subsection{Comparison with Analytical Solution}

For the unit disk, the BEM and analytical solutions should agree:
\begin{equation}
    G_{\text{BEM}}(\vect{x}, \bm{\xi}) \approx G_{\D}(\vect{x}, \bm{\xi})
\end{equation}
with differences due only to quadrature error. This provides a valuable validation tool.


% ============================================================================
% SECTION 5: FINITE ELEMENT METHOD
% ============================================================================
\section{Finite Element Method}
\label{sec:fem}

The Finite Element Method (FEM) provides a systematic approach to discretizing the weak formulation on general domains \citep{brenner2008mathematical,ern2004theory}.

\subsection{Discrete Problem}

Using piecewise linear (P1) Lagrange elements on a triangular mesh $\mathcal{T}_h$:
\begin{equation}
    \label{eq:fem_system}
    \mat{A} \vect{u} = \vect{b}
\end{equation}
where $A_{ij} = \int_\Omega \grad \phi_j \cdot \grad \phi_i \, \dd\vect{x}$ and $b_i = \sum_{k=1}^{N} q_k \, \phi_i(\bm{\xi}_k)$.

\subsection{Load Vector Assembly}

The load vector requires evaluating basis functions at source positions using barycentric coordinates:
\begin{equation}
    \label{eq:interp_method}
    b_i = \sum_{k=1}^{N} q_k \, \lambda_i(\bm{\xi}_k)
\end{equation}

\textbf{Key property}: This makes $\vect{b}$ a \emph{piecewise linear} function of source position, affecting the optimization landscape.


% ============================================================================
% SECTION 6: CONFORMAL MAPPING
% ============================================================================
\section{Conformal Mapping for General Domains}
\label{sec:conformal}

Conformal mapping techniques extend the analytical approach to general simply connected domains \citep{ablowitz2003complex,driscoll2002schwarz}. This section provides rigorous proofs of the key results.

\subsection{Conformal Maps and the Riemann Mapping Theorem}

\begin{definition}[Conformal Map]
A function $f: \Omega \to \D$ is \textbf{conformal} if it is holomorphic (complex-differentiable) and bijective. Conformal maps preserve angles and local shapes.
\end{definition}

\begin{theorem}[Riemann Mapping Theorem]
Any simply connected domain $\Omega \subsetneq \C$ (not the entire complex plane) is conformally equivalent to the unit disk $\D$. That is, there exists a conformal bijection $f: \Omega \to \D$.
\end{theorem}

\subsection{Transformation of the Laplacian}

\begin{theorem}[Conformal Transformation of the Laplacian]
\label{thm:laplacian_transform}
Let $f: \Omega \to \D$ be conformal with $w = f(z)$. Then:
\begin{equation}
    \laplacian_z = |f'(z)|^2 \laplacian_w
\end{equation}
\end{theorem}

\begin{proof}
For a conformal (holomorphic) map $w = f(z)$, the Cauchy-Riemann equations imply $\frac{\pp f}{\pp \bar{z}} = 0$.

The Laplacian in complex coordinates is:
\begin{equation}
    \laplacian = 4\frac{\pp^2}{\pp z \pp \bar{z}}
\end{equation}

Under the coordinate change $w = f(z)$:
\begin{equation}
    \frac{\pp}{\pp z} = f'(z) \frac{\pp}{\pp w}, \quad 
    \frac{\pp}{\pp \bar{z}} = \overline{f'(z)} \frac{\pp}{\pp \bar{w}}
\end{equation}

Therefore:
\begin{equation}
    \laplacian_z = 4\frac{\pp^2}{\pp z \pp \bar{z}} = 4 |f'(z)|^2 \frac{\pp^2}{\pp w \pp \bar{w}} = |f'(z)|^2 \laplacian_w
\end{equation}
\end{proof}

\subsection{Transformation of the Delta Function}

\begin{lemma}[Delta Function Under Coordinate Change]
\label{lem:delta_transform}
Under a smooth bijection $w = f(z)$ with Jacobian $J = |f'(z)|^2$:
\begin{equation}
    \dirac(z - z_k) = |f'(z_k)|^2 \dirac(w - w_k)
\end{equation}
where $w_k = f(z_k)$.
\end{lemma}

\begin{proof}
The delta function transforms according to the change of variables formula. For any test function $\phi$:
\begin{equation}
    \int_\Omega \dirac(z - z_k) \phi(z) \, d^2z = \phi(z_k)
\end{equation}

Under the change $w = f(z)$ with $d^2z = |f'(z)|^{-2} d^2w$:
\begin{equation}
    \int_{\D} \dirac(w - w_k) \phi(f^{-1}(w)) |f'(f^{-1}(w))|^{-2} \, d^2w = \phi(z_k) |f'(z_k)|^{-2}
\end{equation}

Comparing, we obtain: $\dirac(z - z_k) = |f'(z_k)|^2 \dirac(w - w_k)$.
\end{proof}

\subsection{The Main Theorem: Preservation of Source Intensities}

\begin{theorem}[Conformal Invariance of Point Source Problems]
\label{thm:intensity_preservation}
Let $f: \Omega \to \D$ be conformal. If $\tilde{u}(w)$ solves
\begin{equation}
    -\laplacian_w \tilde{u} = \sum_{k=1}^{N} q_k \dirac(w - w_k), \quad \frac{\pp \tilde{u}}{\pp n_w} = 0 \text{ on } \pp\D
\end{equation}
then $u(z) := \tilde{u}(f(z))$ solves
\begin{equation}
    -\laplacian_z u = \sum_{k=1}^{N} q_k \dirac(z - z_k), \quad \frac{\pp u}{\pp n_z} = 0 \text{ on } \boundary
\end{equation}
where $z_k = f^{-1}(w_k)$. \textbf{The intensities $q_k$ are unchanged.}
\end{theorem}

\begin{proof}
Define $u(z) = \tilde{u}(f(z))$. We show $u$ satisfies the PDE in $\Omega$.

\textbf{Step 1: Transform the Laplacian.}
By \cref{thm:laplacian_transform}:
\begin{equation}
    -\laplacian_z u(z) = -|f'(z)|^2 \laplacian_w \tilde{u}(f(z))
\end{equation}

\textbf{Step 2: Substitute the disk equation.}
Since $\tilde{u}$ solves the disk problem:
\begin{equation}
    -\laplacian_z u(z) = |f'(z)|^2 \sum_{k=1}^{N} q_k \dirac(w - w_k)
\end{equation}

\textbf{Step 3: Transform delta functions back.}
Using \cref{lem:delta_transform}, $\dirac(w - w_k) = |f'(z_k)|^{-2} \dirac(z - z_k)$:
\begin{equation}
    -\laplacian_z u(z) = |f'(z)|^2 \sum_{k=1}^{N} \frac{q_k}{|f'(z_k)|^2} \dirac(z - z_k)
\end{equation}

\textbf{Step 4: Apply the key distributional identity.}
For any smooth function $g(z)$ and delta distribution:
\begin{equation}
    \boxed{g(z) \cdot \dirac(z - z_k) = g(z_k) \cdot \dirac(z - z_k)}
\end{equation}
This holds because $\dirac(z - z_k)$ is supported only at $z = z_k$, where it ``samples'' $g$.

Applying this with $g(z) = |f'(z)|^2$:
\begin{equation}
    |f'(z)|^2 \cdot \frac{1}{|f'(z_k)|^2} \dirac(z - z_k) = \frac{|f'(z_k)|^2}{|f'(z_k)|^2} \dirac(z - z_k) = \dirac(z - z_k)
\end{equation}

\textbf{Step 5: Conclude.}
Therefore:
\begin{equation}
    -\laplacian_z u(z) = \sum_{k=1}^{N} q_k \dirac(z - z_k)
\end{equation}
with the \textbf{same} intensities $q_k$.

\textbf{Step 6: Boundary condition.}
On $\boundary$, the normal derivative transforms as:
\begin{equation}
    \frac{\pp u}{\pp n_z} = \frac{1}{|f'(z)|} \frac{\pp \tilde{u}}{\pp n_w}
\end{equation}
Since $\frac{\pp \tilde{u}}{\pp n_w} = 0$ on $\pp\D$, we have $\frac{\pp u}{\pp n_z} = 0$ on $\boundary$.
\end{proof}

\subsection{The Solution Formula for General Domains}

\begin{corollary}[Universal Solution Formula]
\label{cor:general_solution}
For any simply connected domain $\Omega$ with conformal map $f: \Omega \to \D$, the solution to the Poisson-Neumann problem with point sources at $z_k \in \Omega$ and intensities $q_k$ is:
\begin{equation}
    \label{eq:general_solution}
    \boxed{u(z) = \sum_{k=1}^{N} q_k \, G_{\D}(f(z), f(z_k))}
\end{equation}
where $G_{\D}$ is the disk Green's function \eqref{eq:disk_greens}.
\end{corollary}

\begin{remark}[Practical Implications]
This formula is powerful because it requires only:
\begin{enumerate}
    \item The disk Green's function $G_{\D}$ (derived analytically in \cref{sec:analytical})
    \item The conformal map $f: \Omega \to \D$ (explicit or numerical)
\end{enumerate}
No new PDEs need to be solved for each domain type.
\end{remark}

\subsection{Conformal Maps for Specific Domains}

\subsubsection{Ellipse}

For an ellipse with semi-axes $a > b$, the conformal map uses the inverse Joukowsky transformation.

\begin{proposition}[Ellipse Conformal Map]
The ellipse $\{(x,y): (x/a)^2 + (y/b)^2 < 1\}$ is conformally mapped to the unit disk by:
\begin{equation}
    f(z) = \frac{z - \sqrt{z^2 - c^2}}{c}
\end{equation}
where $c = \sqrt{a^2 - b^2}$ is the focal distance, with appropriate branch cut selection.
\end{proposition}

\subsubsection{Polygons: Schwarz-Christoffel Transformation}

For a polygon with vertices $v_1, \ldots, v_n$ and interior angles $\alpha_1, \ldots, \alpha_n$:

\begin{theorem}[Schwarz-Christoffel Formula]
The conformal map from the unit disk to a polygon is:
\begin{equation}
    f^{-1}(w) = A \int_0^w \prod_{k=1}^{n} \left(1 - \frac{\zeta}{w_k}\right)^{\alpha_k/\pi - 1} d\zeta + B
\end{equation}
where $w_k = e^{i\theta_k}$ are the \textbf{prevertices} on the unit circle, and $A, B$ are constants.
\end{theorem}

Finding the prevertices requires solving a nonlinear system (the ``parameter problem'') \citep{driscoll2002schwarz}.

\subsubsection{Brain-like Domains}

For EEG source localization applications, we consider domains resembling 2D brain cross-sections. These can be parameterized as smooth closed curves and mapped to the disk using numerical conformal mapping algorithms.


% ============================================================================
% SECTION 7: INVERSE PROBLEM
% ============================================================================
\section{Inverse Problem Formulations}
\label{sec:inverse}

\subsection{Problem Statement}

Given boundary measurements $u^{\text{meas}}$, find source parameters $\{(\bm{\xi}_k, q_k)\}$ such that $u(\vect{x}) \approx u^{\text{meas}}(\vect{x})$ on $\boundary$, subject to $\sum_k q_k = 0$.

\subsection{Ill-Posedness}

The inverse source problem is ill-posed \citep{hadamard1923lectures,kirsch2011introduction}: non-unique and unstable. Regularization is essential.

\subsection{Two Formulations}

\textbf{Nonlinear}: Optimize positions and intensities continuously:
\begin{equation}
    \min_{\{\bm{\xi}_k, q_k\}} \norm{u(\cdot; \bm{\xi}, \vect{q}) - u^{\text{meas}}}^2
\end{equation}

\textbf{Linear}: Fix positions on a grid, solve for intensities:
\begin{equation}
    \min_{\vect{q}} \norm{\mat{G}\vect{q} - \vect{u}^{\text{meas}}}^2 + \mathcal{R}(\vect{q})
\end{equation}


% ============================================================================
% SECTION 8: REGULARIZATION
% ============================================================================
\section{Regularization Methods}
\label{sec:regularization}

\subsection{Tikhonov ($L^2$)}
\begin{equation}
    \min_{\vect{q}} \norm{\mat{G}\vect{q} - \vect{u}^{\text{meas}}}_2^2 + \alpha \norm{\vect{q}}_2^2
\end{equation}
Closed-form solution; produces smooth (diffuse) reconstructions.

\subsection{Sparsity ($L^1$)}
\begin{equation}
    \min_{\vect{q}} \norm{\mat{G}\vect{q} - \vect{u}^{\text{meas}}}_2^2 + \alpha \norm{\vect{q}}_1
\end{equation}
Promotes sparse solutions \citep{tibshirani1996regression}; better for point sources.

\subsection{Total Variation}
\begin{equation}
    \min_{\vect{q}} \norm{\mat{G}\vect{q} - \vect{u}^{\text{meas}}}_2^2 + \alpha \norm{\mat{D}\vect{q}}_1
\end{equation}
Promotes piecewise constant solutions \citep{rudin1992nonlinear}; solved via ADMM \citep{boyd2011distributed} or Chambolle-Pock \citep{chambolle2011first}.

\subsection{L-Curve Method for Parameter Selection}

The regularization parameter $\alpha$ critically affects solution quality. The \textbf{L-curve method} \citep{hansen1992analysis} plots:
\begin{itemize}
    \item $\rho(\alpha) = \norm{\mat{G}\vect{q}_\alpha - \vect{u}^{\text{meas}}}_2$ (residual)
    \item $\eta(\alpha) = \mathcal{R}(\vect{q}_\alpha)$ (regularization term)
\end{itemize}
in log-log scale. The optimal $\alpha$ is at the ``corner'' of the L-shaped curve.

\begin{remark}[Importance of L-Curve]
Experiments show that heuristic $\alpha$ values can be off by factors of $1000\times$, especially for TV regularization. L-curve adaptation to domain size and noise level is essential.
\end{remark}


% ============================================================================
% SECTION 9: NUMERICAL EXPERIMENTS
% ============================================================================
\section{Numerical Experiments}
\label{sec:numerical}

\subsection{Test Problem}

Four point sources in the unit disk with positions $(-0.3, 0.4)$, $(0.5, 0.3)$, $(-0.4, -0.4)$, $(0.3, -0.5)$ and intensities $\pm 1$.

\subsection{Results Summary}

\begin{table}[H]
\centering
\caption{Solver comparison (Position RMSE / Time)}
\label{tab:results_summary}
\begin{tabular}{lccc}
\toprule
& \textbf{Analytical} & \textbf{BEM} & \textbf{FEM} \\
\midrule
Linear $L^1$ & 0.259 / 0.1s & 0.273 / 0.1s & 0.273 / 0.5s \\
Linear $L^2$ & 0.052 / 0.0s & 0.052 / 0.2s & 0.052 / 0.5s \\
Nonlinear (DE) & 0.235 / 4.9s & 0.269 / 31s & \textbf{0.004} / 58s \\
\bottomrule
\end{tabular}
\end{table}

\subsection{Results with L-Curve Parameter Selection}

\begin{table}[H]
\centering
\caption{Linear solver comparison with L-curve optimal $\alpha$}
\label{tab:lcurve_results}
\begin{tabular}{llcc}
\toprule
\textbf{Domain} & \textbf{Method} & \textbf{Loc. Score} & \textbf{$\alpha$ (L-curve)} \\
\midrule
Disk & L2 & 0.37 & $2.3 \times 10^{-4}$ \\
     & TV & 0.38 & $3.8 \times 10^{-5}$ \\
\midrule
Ellipse & L2 & 0.62 & $2.3 \times 10^{-4}$ \\
        & TV & 0.71 & $7.0 \times 10^{-5}$ \\
\midrule
Square & L2 & 0.34 & $7.9 \times 10^{-4}$ \\
       & TV & 0.35 & $1.3 \times 10^{-4}$ \\
\bottomrule
\end{tabular}
\end{table}

\subsection{Key Finding: Optimization Landscape}

The FEM nonlinear solver achieved near-perfect recovery (RMSE = 0.004), far outperforming analytical and BEM versions. This is due to FEM's \textbf{smoother optimization landscape}:
\begin{itemize}
    \item FEM's piecewise linear interpolation creates broader valleys
    \item The mesh discretization limits sharp gradients
    \item Fewer narrow local minima trap the optimizer
\end{itemize}

Analytical and BEM have equivalent landscapes (same Green's function), so their performance differences are due to random variation in restarts.


% ============================================================================
% SECTION 10: IMPLEMENTATION
% ============================================================================
\section{Implementation}
\label{sec:implementation}

\subsection{Software Modules}

\begin{itemize}
    \item \texttt{analytical\_solver.py}: Exact Green's function
    \item \texttt{bem\_solver.py}: Numerical BEM
    \item \texttt{fem\_solver.py}: FEM via scikit-fem \citep{gustafsson2020scikit}
    \item \texttt{conformal\_solver.py}: Conformal mapping framework
    \item \texttt{mesh.py}: Domain meshes (disk, ellipse, polygon, brain)
    \item \texttt{parameter\_selection.py}: L-curve analysis
    \item \texttt{regularization.py}: $L^1$, $L^2$, TV
    \item \texttt{comparison.py}: Comprehensive benchmarks
    \item \texttt{cli.py}: Command-line interface
\end{itemize}

\subsection{Usage}

\begin{lstlisting}[language=bash]
# Compare solvers on brain domain with L-curve
python -m inverse_source.cli compare --domain brain --alpha auto

# Ellipse with specific parameters
python -m inverse_source.cli compare --domain ellipse --ellipse-a 2.0 --ellipse-b 1.0
\end{lstlisting}

\subsection{Dependencies}

NumPy \citep{harris2020array}, SciPy \citep{virtanen2020scipy}, scikit-fem \citep{gustafsson2020scikit}, CVXPY, Matplotlib.


% ============================================================================
% SECTION 11: CONCLUSIONS
% ============================================================================
\section{Conclusions}
\label{sec:conclusions}

This report presented mathematical foundations and numerical methods for inverse source localization. Key contributions:

\begin{enumerate}
    \item \textbf{Rigorous conformal mapping theory}: We proved that source intensities are preserved under conformal transformation (\cref{thm:intensity_preservation}), enabling the universal solution formula:
    \begin{equation}
        u(z) = \sum_k q_k \, G_{\D}(f(z), f(z_k))
    \end{equation}
    
    \item \textbf{Complete Green's function derivation}: Method of images for the disk with full proofs (\cref{thm:disk_greens})
    
    \item Three forward solvers (Analytical, BEM, FEM) produce equivalent results for linear inverse problems
    
    \item For nonlinear problems, FEM's smoother landscape enables better global optimization
    
    \item \textbf{L-curve parameter selection}: Essential for reliable regularization across domains
    
    \item Conformal mapping extends analytical methods to ellipses, polygons, and brain-like domains
\end{enumerate}

\subsection{Future Work}

\begin{itemize}
    \item Extend Schwarz-Christoffel implementation for arbitrary polygons
    \item Apply to real EEG data
    \item Investigate 3D extensions (where conformal mapping is more restrictive)
    \item Bayesian uncertainty quantification
\end{itemize}


% ============================================================================
% BIBLIOGRAPHY
% ============================================================================
\newpage
\bibliographystyle{abbrvnat}
\bibliography{references}

\end{document}
