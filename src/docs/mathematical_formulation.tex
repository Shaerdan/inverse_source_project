\documentclass[11pt]{article}
\usepackage[margin=1in]{geometry}
\usepackage{amsmath,amssymb,amsthm}
\usepackage{bm}
\usepackage{hyperref}

\newtheorem{definition}{Definition}
\newtheorem{theorem}{Theorem}
\newtheorem{remark}{Remark}

\title{Inverse Source Localization: Complete Mathematical Formulation}
\author{Mathematical Reference Document}
\date{}

\begin{document}
\maketitle

\tableofcontents
\newpage

%==============================================================================
\section{Problem Statement}
%==============================================================================

\subsection{Physical Setup}

We consider the steady-state potential field $u: \Omega \to \mathbb{R}$ in a bounded domain $\Omega \subset \mathbb{R}^2$ with boundary $\partial\Omega$, generated by $K$ point sources located at positions $\bm{z}_k \in \Omega$ with intensities $q_k \in \mathbb{R}$.

\subsection{Governing Equation}

The potential satisfies the Poisson equation with point source terms:
\begin{equation}
    -\Delta u(\bm{x}) = \sum_{k=1}^{K} q_k \, \delta(\bm{x} - \bm{z}_k), \quad \bm{x} \in \Omega
\end{equation}
where $\delta(\cdot)$ is the Dirac delta distribution.

\subsection{Boundary Conditions}

We impose homogeneous Neumann (no-flux) boundary conditions:
\begin{equation}
    \frac{\partial u}{\partial n}(\bm{x}) = 0, \quad \bm{x} \in \partial\Omega
\end{equation}
where $\bm{n}$ is the outward unit normal to $\partial\Omega$.

\subsection{Compatibility Condition}

For the Neumann problem to have a solution, the source terms must satisfy:
\begin{equation}
    \sum_{k=1}^{K} q_k = 0
\end{equation}

\begin{proof}
Integrating the governing equation over $\Omega$ and applying the divergence theorem:
\begin{align}
    \int_\Omega (-\Delta u) \, d\bm{x} &= \sum_{k=1}^{K} q_k \int_\Omega \delta(\bm{x} - \bm{z}_k) \, d\bm{x} \\
    -\int_{\partial\Omega} \frac{\partial u}{\partial n} \, ds &= \sum_{k=1}^{K} q_k \\
    0 &= \sum_{k=1}^{K} q_k
\end{align}
\end{proof}

\subsection{Uniqueness}

The solution $u$ is unique only up to an additive constant. We typically impose:
\begin{equation}
    \int_{\partial\Omega} u \, ds = 0 \quad \text{or} \quad u(\bm{x}_0) = 0 \text{ for some reference point } \bm{x}_0
\end{equation}

%==============================================================================
\section{Forward Problem: Analytical Solution (Unit Disk)}
%==============================================================================

\subsection{Domain}

Let $\Omega = \{\bm{x} \in \mathbb{R}^2 : |\bm{x}| < 1\}$ be the open unit disk with boundary $\partial\Omega = \{\bm{x} : |\bm{x}| = 1\}$.

\subsection{Neumann Green's Function}

\begin{definition}[Neumann Green's Function]
The Neumann Green's function $G_N(\bm{x}, \bm{\xi})$ for the unit disk satisfies:
\begin{align}
    -\Delta_{\bm{x}} G_N(\bm{x}, \bm{\xi}) &= \delta(\bm{x} - \bm{\xi}) - \frac{1}{|\Omega|}, \quad \bm{x} \in \Omega \\
    \frac{\partial G_N}{\partial n_{\bm{x}}}(\bm{x}, \bm{\xi}) &= 0, \quad \bm{x} \in \partial\Omega
\end{align}
where $|\Omega| = \pi$ is the area of the unit disk, and the term $-1/|\Omega|$ ensures compatibility.
\end{definition}

\subsection{Derivation via Method of Images}

\subsubsection{Free-Space Green's Function}

The fundamental solution to $-\Delta G = \delta$ in $\mathbb{R}^2$ is:
\begin{equation}
    G_0(\bm{x}, \bm{\xi}) = -\frac{1}{2\pi} \ln|\bm{x} - \bm{\xi}|
\end{equation}

\subsubsection{Image Point}

For a source at $\bm{\xi} \in \Omega$ (with $|\bm{\xi}| < 1$, $\bm{\xi} \neq \bm{0}$), the image point under inversion in the unit circle is:
\begin{equation}
    \bm{\xi}^* = \frac{\bm{\xi}}{|\bm{\xi}|^2}
\end{equation}

Note: $|\bm{\xi}^*| = 1/|\bm{\xi}| > 1$, so the image lies outside $\Omega$.

\subsubsection{Construction}

The Neumann Green's function is:
\begin{equation}
    G_N(\bm{x}, \bm{\xi}) = -\frac{1}{2\pi} \ln|\bm{x} - \bm{\xi}| - \frac{1}{2\pi} \ln|\bm{x} - \bm{\xi}^*| + \frac{1}{2\pi} \ln|\bm{\xi}| + C
\end{equation}

where the constant $C$ is chosen to satisfy normalization.

\subsubsection{Verification of Neumann Condition}

Using complex notation with $z = x_1 + ix_2$ and $\zeta = \xi_1 + i\xi_2$:
\begin{equation}
    G_N(z, \zeta) = -\frac{1}{2\pi} \ln|z - \zeta| - \frac{1}{2\pi} \ln\left|z - \frac{\bar{\zeta}}{|\zeta|^2}\right| + \frac{1}{2\pi} \ln|\zeta|
\end{equation}

For $|z| = 1$ (boundary), the image term can be rewritten:
\begin{align}
    \left|z - \frac{\bar{\zeta}}{|\zeta|^2}\right| &= \left|\frac{z|\zeta|^2 - \bar{\zeta}}{|\zeta|^2}\right| \\
    &= \frac{1}{|\zeta|} |z|\zeta|^2 - \bar{\zeta}| \\
    &= \frac{|\zeta|}{|\zeta|} |z - \bar{\zeta}/|\zeta|^2| \cdot |\zeta| \\
    &= |\zeta| \cdot |z\bar{z} - z\bar{\zeta}/|\zeta|^2| \quad \text{(using } |z|^2 = 1\text{)}
\end{align}

After computation, the normal derivative vanishes on $|z| = 1$.

\subsection{Final Formula}

For $z, \zeta \in \Omega$ (unit disk) with $\zeta \neq 0$:
\begin{equation}
    \boxed{G_N(z, \zeta) = -\frac{1}{2\pi} \ln|z - \zeta| - \frac{1}{2\pi} \ln|1 - z\bar{\zeta}| + C}
\end{equation}

where we used: $|z - \bar{\zeta}/|\zeta|^2| = |1 - z\bar{\zeta}|/|\zeta|$ for $|z| = 1$.

\subsection{Solution Formula}

The solution to the forward problem is:
\begin{equation}
    u(\bm{x}) = \sum_{k=1}^{K} q_k \, G_N(\bm{x}, \bm{z}_k)
\end{equation}

At boundary point $\bm{x} \in \partial\Omega$:
\begin{equation}
    u(\bm{x}) = \sum_{k=1}^{K} q_k \left[ -\frac{1}{2\pi} \ln|\bm{x} - \bm{z}_k| - \frac{1}{2\pi} \ln|1 - \bm{x} \cdot \bar{\bm{z}}_k| \right]
\end{equation}

%==============================================================================
\section{Forward Problem: Conformal Mapping (General Domains)}
%==============================================================================

\subsection{Principle}

For a simply connected domain $\Omega \subset \mathbb{C}$, the Riemann Mapping Theorem guarantees existence of a conformal (holomorphic, bijective) map:
\begin{equation}
    f: \Omega \to \mathbb{D}
\end{equation}
where $\mathbb{D} = \{w \in \mathbb{C} : |w| < 1\}$ is the unit disk.

\subsection{Key Property}

The Laplacian transforms under conformal mapping as:
\begin{equation}
    \Delta_z u = |f'(z)|^2 \Delta_w (u \circ f^{-1})
\end{equation}

This means harmonic functions map to harmonic functions, and the Neumann Green's function transforms as:
\begin{equation}
    G_N^\Omega(z_1, z_2) = G_N^{\mathbb{D}}(f(z_1), f(z_2))
\end{equation}

\subsection{Solution Formula for General Domain}

Given sources at $z_k \in \Omega$ with intensities $q_k$:
\begin{equation}
    \boxed{u(z) = \sum_{k=1}^{K} q_k \, G_N^{\mathbb{D}}(f(z), f(z_k))}
\end{equation}

where $G_N^{\mathbb{D}}$ is the unit disk Neumann Green's function from Section 2.

\subsection{Implemented Conformal Maps}

\subsubsection{Ellipse: $\{(x,y) : x^2/a^2 + y^2/b^2 < 1\}$}

Using the Joukowsky map and its inverse:
\begin{equation}
    f^{-1}(w) = \frac{a+b}{2} w + \frac{a-b}{2} \frac{1}{w}
\end{equation}

Maps unit disk to ellipse with semi-axes $a, b$.

\subsubsection{Rectangle: $(-a, a) \times (-b, b)$}

Uses Schwarz-Christoffel transformation involving elliptic integrals:
\begin{equation}
    f^{-1}(w) = C \int_0^w \frac{d\zeta}{\sqrt{(1-\zeta^2)(1-k^2\zeta^2)}}
\end{equation}
where $k$ is the elliptic modulus determined by aspect ratio.

\subsubsection{Polygon (General)}

Schwarz-Christoffel formula:
\begin{equation}
    f^{-1}(w) = C_1 + C_2 \int_0^w \prod_{j=1}^{n} (\zeta - w_j)^{\alpha_j - 1} d\zeta
\end{equation}
where $w_j$ are prevertices on $\partial\mathbb{D}$ and $\alpha_j \pi$ are interior angles.

\subsubsection{Star Domain: $r(\theta) = 1 + A\cos(n\theta)$}

Computed numerically via boundary correspondence and Fourier series.

\subsubsection{Brain-like Domain}

Numerical conformal map computed from boundary parameterization.

%==============================================================================
\section{Forward Problem: Finite Element Method}
%==============================================================================

\subsection{Weak Formulation}

Multiply the PDE by test function $v \in H^1(\Omega)$ and integrate:
\begin{equation}
    -\int_\Omega (\Delta u) v \, d\bm{x} = \sum_{k=1}^{K} q_k \int_\Omega \delta(\bm{x} - \bm{z}_k) v(\bm{x}) \, d\bm{x}
\end{equation}

Apply Green's first identity:
\begin{equation}
    \int_\Omega \nabla u \cdot \nabla v \, d\bm{x} - \int_{\partial\Omega} \frac{\partial u}{\partial n} v \, ds = \sum_{k=1}^{K} q_k \, v(\bm{z}_k)
\end{equation}

With homogeneous Neumann BC, the boundary integral vanishes:
\begin{equation}
    \boxed{\int_\Omega \nabla u \cdot \nabla v \, d\bm{x} = \sum_{k=1}^{K} q_k \, v(\bm{z}_k)}
\end{equation}

\subsection{Finite Element Discretization}

\subsubsection{Mesh and Basis Functions}

Triangulate $\Omega$ into elements with $N$ nodes. Define piecewise linear (P1) basis functions $\{\phi_i\}_{i=1}^{N}$ satisfying:
\begin{equation}
    \phi_i(\bm{x}_j) = \delta_{ij} \quad \text{(Kronecker delta)}
\end{equation}

\subsubsection{Discrete Solution}

Approximate:
\begin{equation}
    u_h(\bm{x}) = \sum_{j=1}^{N} u_j \phi_j(\bm{x})
\end{equation}

\subsubsection{Stiffness Matrix}

Taking $v = \phi_i$ for each $i$:
\begin{equation}
    \sum_{j=1}^{N} u_j \int_\Omega \nabla \phi_j \cdot \nabla \phi_i \, d\bm{x} = \sum_{k=1}^{K} q_k \, \phi_i(\bm{z}_k)
\end{equation}

Define the stiffness matrix $\bm{K} \in \mathbb{R}^{N \times N}$:
\begin{equation}
    K_{ij} = \int_\Omega \nabla \phi_i \cdot \nabla \phi_j \, d\bm{x}
\end{equation}

\subsubsection{Load Vector}

Define load vector $\bm{f} \in \mathbb{R}^N$:
\begin{equation}
    f_i = \sum_{k=1}^{K} q_k \, \phi_i(\bm{z}_k)
\end{equation}

\textbf{Case 1: Source at mesh node $\bm{z}_k = \bm{x}_j$}
\begin{equation}
    f_i = \sum_{k=1}^{K} q_k \, \delta_{i,j_k} = q_k \text{ if node } i \text{ is a source location, else } 0
\end{equation}

\textbf{Case 2: Source at arbitrary point $\bm{z}_k$}

If $\bm{z}_k$ lies in element $T$ with vertices $\bm{x}_{i_1}, \bm{x}_{i_2}, \bm{x}_{i_3}$:
\begin{equation}
    \phi_{i_m}(\bm{z}_k) = \lambda_m(\bm{z}_k), \quad m = 1,2,3
\end{equation}
where $\lambda_m$ are barycentric coordinates of $\bm{z}_k$ in $T$:
\begin{equation}
    \bm{z}_k = \lambda_1 \bm{x}_{i_1} + \lambda_2 \bm{x}_{i_2} + \lambda_3 \bm{x}_{i_3}, \quad \lambda_1 + \lambda_2 + \lambda_3 = 1
\end{equation}

\subsubsection{Linear System}

\begin{equation}
    \bm{K} \bm{u} = \bm{f}
\end{equation}

\textbf{Singularity:} $\bm{K}$ is singular (constant functions in null space). We fix this by:
\begin{itemize}
    \item Pinning one node: set $u_1 = 0$ (modify row 1 of $\bm{K}$)
    \item Or: remove constant mode via constraint $\sum_i u_i = 0$
\end{itemize}

\subsection{Extracting Boundary Values}

Let $\mathcal{B} \subset \{1, \ldots, N\}$ be indices of boundary nodes. The boundary measurements are:
\begin{equation}
    \bm{u}_{\text{boundary}} = \bm{u}|_{\mathcal{B}} \in \mathbb{R}^{|\mathcal{B}|}
\end{equation}

%==============================================================================
\section{Inverse Problem: Linear Formulation (Distributed Sources)}
%==============================================================================

\subsection{Setup}

\subsubsection{Measurement}

Given noisy boundary measurements:
\begin{equation}
    \bm{u}^{\text{meas}} = \bm{u}^{\text{true}} + \bm{\eta}, \quad \bm{\eta} \sim \mathcal{N}(0, \sigma^2 \bm{I})
\end{equation}

\subsubsection{Source Grid}

Fix $M$ candidate source locations $\{\bm{\xi}_j\}_{j=1}^{M}$ in $\Omega$.

\subsubsection{Green's Matrix}

Define $\bm{G} \in \mathbb{R}^{N_b \times M}$ where $N_b$ is number of boundary measurement points:
\begin{equation}
    G_{ij} = G_N(\bm{x}_i^{\text{boundary}}, \bm{\xi}_j)
\end{equation}

This is precomputed by solving $M$ forward problems, each with unit source at $\bm{\xi}_j$.

\subsubsection{Forward Model}

\begin{equation}
    \bm{u}_{\text{boundary}} = \bm{G} \bm{q}
\end{equation}
where $\bm{q} \in \mathbb{R}^M$ is the vector of source intensities.

\subsection{Compatibility Constraint}

From Section 1.4:
\begin{equation}
    \sum_{j=1}^{M} q_j = 0 \quad \Leftrightarrow \quad \bm{1}^\top \bm{q} = 0
\end{equation}

\subsection{L2 Regularization (Tikhonov)}

\subsubsection{Optimization Problem}

\begin{equation}
    \min_{\bm{q} \in \mathbb{R}^M} \frac{1}{2} \|\bm{G}\bm{q} - \tilde{\bm{u}}\|_2^2 + \frac{\alpha}{2} \|\bm{q}\|_2^2 \quad \text{subject to} \quad \bm{1}^\top \bm{q} = 0
\end{equation}

where $\tilde{\bm{u}} = \bm{u}^{\text{meas}} - \text{mean}(\bm{u}^{\text{meas}})$ (centered measurements).

\subsubsection{Lagrangian}

\begin{equation}
    \mathcal{L}(\bm{q}, \lambda) = \frac{1}{2} \|\bm{G}\bm{q} - \tilde{\bm{u}}\|_2^2 + \frac{\alpha}{2} \|\bm{q}\|_2^2 + \lambda \bm{1}^\top \bm{q}
\end{equation}

\subsubsection{Optimality Conditions}

\begin{align}
    \nabla_{\bm{q}} \mathcal{L} &= \bm{G}^\top(\bm{G}\bm{q} - \tilde{\bm{u}}) + \alpha \bm{q} + \lambda \bm{1} = \bm{0} \\
    \nabla_\lambda \mathcal{L} &= \bm{1}^\top \bm{q} = 0
\end{align}

\subsubsection{Solution}

From the first equation:
\begin{equation}
    (\bm{G}^\top \bm{G} + \alpha \bm{I}) \bm{q} = \bm{G}^\top \tilde{\bm{u}} - \lambda \bm{1}
\end{equation}

Let $\bm{A} = \bm{G}^\top \bm{G} + \alpha \bm{I}$ (symmetric positive definite). Then:
\begin{equation}
    \bm{q} = \bm{A}^{-1}(\bm{G}^\top \tilde{\bm{u}} - \lambda \bm{1})
\end{equation}

Applying constraint $\bm{1}^\top \bm{q} = 0$:
\begin{equation}
    \bm{1}^\top \bm{A}^{-1} \bm{G}^\top \tilde{\bm{u}} = \lambda \bm{1}^\top \bm{A}^{-1} \bm{1}
\end{equation}

\begin{equation}
    \boxed{\lambda = \frac{\bm{1}^\top \bm{A}^{-1} \bm{G}^\top \tilde{\bm{u}}}{\bm{1}^\top \bm{A}^{-1} \bm{1}}}
\end{equation}

\begin{equation}
    \boxed{\bm{q}^* = \bm{A}^{-1}\left(\bm{G}^\top \tilde{\bm{u}} - \lambda^* \bm{1}\right)}
\end{equation}

\subsection{L1 Regularization (Sparsity-Promoting)}

\subsubsection{Optimization Problem}

\begin{equation}
    \min_{\bm{q} \in \mathbb{R}^M} \frac{1}{2} \|\bm{G}\bm{q} - \tilde{\bm{u}}\|_2^2 + \alpha \|\bm{q}\|_1 \quad \text{subject to} \quad \bm{1}^\top \bm{q} = 0
\end{equation}

where $\|\bm{q}\|_1 = \sum_{j=1}^{M} |q_j|$.

\subsubsection{Convex Program Formulation}

This is a convex optimization problem. We solve it using CVXPY:

\begin{verbatim}
import cvxpy as cp
q = cp.Variable(M)
objective = 0.5 * cp.sum_squares(G @ q - u_tilde) + alpha * cp.norm1(q)
constraints = [cp.sum(q) == 0]
problem = cp.Problem(cp.Minimize(objective), constraints)
problem.solve()
\end{verbatim}

\subsubsection{Properties}

\begin{itemize}
    \item L1 promotes sparsity: many $q_j$ will be exactly zero
    \item Non-differentiable at $q_j = 0$, requires specialized solvers
    \item Tends to concentrate intensity at fewer grid points
\end{itemize}

\subsection{Total Variation (TV) Regularization}

\subsubsection{Discrete Gradient Operator}

For grid points $\{\bm{\xi}_j\}$, define a gradient operator $\bm{D} \in \mathbb{R}^{E \times M}$ where $E$ is the number of edges connecting neighboring points.

For each edge $(j, k)$ connecting $\bm{\xi}_j$ and $\bm{\xi}_k$:
\begin{equation}
    (\bm{D}\bm{q})_{jk} = \frac{q_k - q_j}{\|\bm{\xi}_k - \bm{\xi}_j\|}
\end{equation}

In matrix form, row corresponding to edge $(j,k)$:
\begin{equation}
    D_{\text{row}, j} = -\frac{1}{\|\bm{\xi}_k - \bm{\xi}_j\|}, \quad D_{\text{row}, k} = +\frac{1}{\|\bm{\xi}_k - \bm{\xi}_j\|}
\end{equation}

\subsubsection{Neighborhood Construction}

We use a KD-tree to find neighbors within radius $r$:
\begin{equation}
    \mathcal{N}_j = \{k : \|\bm{\xi}_k - \bm{\xi}_j\| < r, \, k \neq j\}
\end{equation}

\subsubsection{Optimization Problem}

\begin{equation}
    \min_{\bm{q} \in \mathbb{R}^M} \frac{1}{2} \|\bm{G}\bm{q} - \tilde{\bm{u}}\|_2^2 + \alpha \|\bm{D}\bm{q}\|_1 \quad \text{subject to} \quad \bm{1}^\top \bm{q} = 0
\end{equation}

\subsubsection{Properties}

\begin{itemize}
    \item TV promotes piecewise constant solutions
    \item Preserves sharp edges/jumps in intensity field
    \item Good for localizing distinct sources
\end{itemize}

\subsection{Implementation Notes}

\subsubsection{Centering}

We center both measurements and predictions:
\begin{align}
    \tilde{\bm{u}} &= \bm{u}^{\text{meas}} - \bar{u}^{\text{meas}} \bm{1} \\
    \text{where } \bar{u}^{\text{meas}} &= \frac{1}{N_b} \sum_{i=1}^{N_b} u_i^{\text{meas}}
\end{align}

This removes the arbitrary constant in the Neumann problem.

\subsubsection{Regularization Parameter Selection}

We use L-curve analysis:
\begin{enumerate}
    \item Solve for range of $\alpha$ values
    \item Plot: $\log\|\bm{G}\bm{q} - \tilde{\bm{u}}\|_2$ vs $\log\|\bm{q}\|$ (or $\|\bm{D}\bm{q}\|_1$)
    \item Find corner of L-curve (maximum curvature)
\end{enumerate}

Corner curvature formula:
\begin{equation}
    \kappa = \frac{|\rho'' \eta' - \rho' \eta''|}{((\rho')^2 + (\eta')^2)^{3/2}}
\end{equation}
where $\rho = \log(\text{residual})$, $\eta = \log(\text{regularizer})$.

%==============================================================================
\section{Inverse Problem: Nonlinear Formulation (Continuous Source Positions)}
%==============================================================================

\subsection{Setup}

\subsubsection{Unknown Parameters}

For $K$ sources (number specified a priori):
\begin{equation}
    \bm{\theta} = (x_1, y_1, q_1, x_2, y_2, q_2, \ldots, x_K, y_K, q_K) \in \mathbb{R}^{3K}
\end{equation}

\subsubsection{Constraints}

\begin{itemize}
    \item Position constraints: $(x_k, y_k) \in \Omega$ for all $k$
    \item Intensity constraint: $\sum_{k=1}^{K} q_k = 0$
\end{itemize}

\subsection{Objective Function}

\begin{equation}
    J(\bm{\theta}) = \frac{1}{2} \|\bm{u}^{\text{forward}}(\bm{\theta}) - \bm{u}^{\text{meas}}\|_2^2
\end{equation}

where $\bm{u}^{\text{forward}}(\bm{\theta})$ is the boundary potential computed by solving the forward problem with sources at positions $(x_k, y_k)$ with intensities $q_k$.

\subsection{Forward Model: Analytical}

For domains with known Green's function:
\begin{equation}
    u^{\text{forward}}(\bm{x}_i; \bm{\theta}) = \sum_{k=1}^{K} q_k \, G_N(\bm{x}_i, (x_k, y_k))
\end{equation}

Gradient with respect to parameters:
\begin{align}
    \frac{\partial u^{\text{forward}}}{\partial q_k} &= G_N(\bm{x}_i, (x_k, y_k)) \\
    \frac{\partial u^{\text{forward}}}{\partial x_k} &= q_k \frac{\partial G_N}{\partial \xi_1}(\bm{x}_i, (x_k, y_k)) \\
    \frac{\partial u^{\text{forward}}}{\partial y_k} &= q_k \frac{\partial G_N}{\partial \xi_2}(\bm{x}_i, (x_k, y_k))
\end{align}

\subsection{Forward Model: FEM}

For each evaluation of $\bm{u}^{\text{forward}}(\bm{\theta})$:

\begin{enumerate}
    \item Build load vector:
    \begin{equation}
        f_i = \sum_{k=1}^{K} q_k \, \phi_i(x_k, y_k)
    \end{equation}
    
    \item Solve linear system:
    \begin{equation}
        \bm{K} \bm{u} = \bm{f}
    \end{equation}
    
    \item Extract boundary values:
    \begin{equation}
        \bm{u}^{\text{forward}} = \bm{u}|_{\mathcal{B}}
    \end{equation}
\end{enumerate}

\subsubsection{Source of Nonlinearity}

The mapping $\bm{\theta} \mapsto \bm{f} \mapsto \bm{u}^{\text{forward}}$ is nonlinear because:
\begin{enumerate}
    \item $\phi_i(x_k, y_k)$ depends nonlinearly on position (piecewise linear basis)
    \item Product $q_k \cdot \phi_i(x_k, y_k)$ couples intensity and position
\end{enumerate}

\subsection{Optimization Methods}

\subsubsection{L-BFGS-B (Local, Gradient-Based)}

\begin{itemize}
    \item Quasi-Newton method with box constraints
    \item Uses approximate Hessian from gradient history
    \item Fast but may converge to local minima
    \item We run multiple restarts from random initializations
\end{itemize}

\subsubsection{Differential Evolution (Global)}

\begin{itemize}
    \item Population-based stochastic optimizer
    \item No gradients needed (derivative-free)
    \item Better at escaping local minima
    \item More expensive (many forward solves)
\end{itemize}

\subsubsection{Handling Constraints}

\textbf{Position constraints:} Enforced via box bounds:
\begin{equation}
    x_{\min} \leq x_k \leq x_{\max}, \quad y_{\min} \leq y_k \leq y_{\max}
\end{equation}

\textbf{Zero-sum intensity constraint:} Either:
\begin{itemize}
    \item Penalty method: add $\mu(\sum_k q_k)^2$ to objective
    \item Elimination: set $q_K = -\sum_{k=1}^{K-1} q_k$, optimize only $q_1, \ldots, q_{K-1}$
\end{itemize}

\subsection{Comparison: Linear vs Nonlinear}

\begin{center}
\begin{tabular}{|l|c|c|}
\hline
\textbf{Aspect} & \textbf{Linear} & \textbf{Nonlinear} \\
\hline
Source positions & Fixed grid & Continuous \\
Number of sources & Discovered via sparsity & Specified a priori \\
Unknowns & $M$ (intensities) & $3K$ (positions + intensities) \\
Optimization & Convex & Non-convex \\
Computation & One-time $\bm{G}$, fast solve & Forward solve per iteration \\
Local minima & No & Yes \\
Resolution & Limited by grid & Arbitrary precision \\
\hline
\end{tabular}
\end{center}

%==============================================================================
\section{Complete Algorithm Summary}
%==============================================================================

\subsection{Forward Problem}

\textbf{Input:} Domain $\Omega$, source positions $\{(x_k, y_k)\}$, intensities $\{q_k\}$

\textbf{Output:} Boundary potential $\bm{u}_{\text{boundary}}$

\begin{enumerate}
    \item \textbf{Analytical (Unit Disk):}
    \begin{equation}
        u(\bm{x}) = \sum_{k} q_k \left[-\frac{1}{2\pi}\ln|\bm{x} - \bm{z}_k| - \frac{1}{2\pi}\ln|1 - \bm{x}\bar{\bm{z}}_k|\right]
    \end{equation}
    
    \item \textbf{Conformal (General Simply Connected):}
    \begin{equation}
        u(z) = \sum_{k} q_k \, G_N^{\mathbb{D}}(f(z), f(z_k))
    \end{equation}
    
    \item \textbf{FEM (Any Domain):}
    \begin{equation}
        \bm{K}\bm{u} = \bm{f}, \quad f_i = \sum_k q_k \phi_i(x_k, y_k)
    \end{equation}
\end{enumerate}

\subsection{Inverse Problem: Linear}

\textbf{Input:} Boundary measurements $\bm{u}^{\text{meas}}$, regularization $\alpha$, method (L1/L2/TV)

\textbf{Output:} Intensity vector $\bm{q}^*$ on grid

\begin{enumerate}
    \item Precompute Green's matrix $\bm{G}$
    \item Center measurements: $\tilde{\bm{u}} = \bm{u}^{\text{meas}} - \bar{u}$
    \item Solve:
    \begin{equation}
        \bm{q}^* = \arg\min_{\bm{1}^\top\bm{q}=0} \frac{1}{2}\|\bm{G}\bm{q} - \tilde{\bm{u}}\|_2^2 + \alpha \, R(\bm{q})
    \end{equation}
    where $R(\bm{q}) = \|\bm{q}\|_2^2$ (L2), $\|\bm{q}\|_1$ (L1), or $\|\bm{D}\bm{q}\|_1$ (TV)
\end{enumerate}

\subsection{Inverse Problem: Nonlinear}

\textbf{Input:} Boundary measurements $\bm{u}^{\text{meas}}$, number of sources $K$

\textbf{Output:} Estimated positions and intensities $\{(x_k^*, y_k^*, q_k^*)\}$

\begin{enumerate}
    \item Initialize: random or grid-based starting points
    \item Optimize:
    \begin{equation}
        \bm{\theta}^* = \arg\min_{\bm{\theta}} \|\bm{u}^{\text{forward}}(\bm{\theta}) - \bm{u}^{\text{meas}}\|_2^2
    \end{equation}
    subject to $(x_k, y_k) \in \Omega$ and $\sum_k q_k = 0$
    \item Use L-BFGS-B (with restarts) or differential evolution
\end{enumerate}

%==============================================================================
\section{Evaluation Metrics}
%==============================================================================

\subsection{Position RMSE}

For recovered sources matched to true sources via Hungarian algorithm:
\begin{equation}
    \text{RMSE}_{\text{pos}} = \sqrt{\frac{1}{K}\sum_{k=1}^{K} \|\bm{z}_k^{\text{true}} - \bm{z}_{\pi(k)}^{\text{rec}}\|^2}
\end{equation}

\subsection{Intensity RMSE}

\begin{equation}
    \text{RMSE}_{\text{int}} = \sqrt{\frac{1}{K}\sum_{k=1}^{K} (q_k^{\text{true}} - q_{\pi(k)}^{\text{rec}})^2}
\end{equation}

\subsection{Boundary Residual}

\begin{equation}
    \text{Residual} = \frac{\|\bm{G}\bm{q}^{\text{rec}} - \bm{u}^{\text{meas}}\|_2}{\|\bm{u}^{\text{meas}}\|_2}
\end{equation}

\subsection{Localization Score (for Linear Methods)}

Weighted proximity of intensity mass to true sources:
\begin{equation}
    \text{Loc} = \frac{\sum_j |q_j| \cdot \exp(-d_j^2 / 2\sigma^2)}{\sum_j |q_j|}
\end{equation}
where $d_j = \min_k \|\bm{\xi}_j - \bm{z}_k^{\text{true}}\|$.

\subsection{Sparsity Ratio}

\begin{equation}
    \text{Sparsity} = \frac{\#\{j : |q_j| > \epsilon \cdot \max|q|\}}{K^{\text{true}}}
\end{equation}

Ideal value is 1.0 (exactly $K$ nonzero sources).

\end{document}
